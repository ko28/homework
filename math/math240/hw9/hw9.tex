\documentclass[11pt]{scrartcl}
\usepackage[sexy]{../../evan}
\usepackage{cmbright}
\usepackage{cancel}
\usepackage[T1]{fontenc}
%\usepackage{enumerate}
\usepackage[shortlabels]{enumitem}
\usepackage[utf8]{inputenc}
\usepackage[margin=1in]{geometry}
%\usepackage[pdfborder={0 0 0},colorlinks=true,citecolor=red{}]{hyperref}
\usepackage{amsmath}
\usepackage{amssymb}
\usepackage{setspace}
\usepackage{systeme}
\usepackage{verbatim}

\title{Math 240: Homework 9}
\author{Daniel Ko}
\date{Spring 2020}

\begin{document}
\maketitle

\section{}
Let $A$ be any nonempty set. (You do not get to decide what $A$ is)
\begin{enumerate}[label=\alph*.]
	\item{
	      Let $R$ be a relation on $A$. Explain what is wrong with the following
	      “proof” that if $R$ is symmetric and transitive, then it is reflexive.\par

	      "Proof": Let $a \in A$. Take an element $b \in A$ such that $(a, b) \in R .$ Because $R$ is symmetric,
	      we also have $(b, a) \in R$. Now using the transitive property, we can conclude that $(a, a) \in R$ because
	      $(a, b) \in R$ and $(b, a) \in R$
	      \begin{proof}
		      This proof is wrong because it assumes a priori that $(a, b) \in R$ when no such definition of the relation $R$ is
		      defined.
	      \end{proof}
	      }
	\item{
	      Give an example of a relation $R$ on $A$ which is symmetric and
	      transitive, but not reflexive. Justify your answer.
	      \begin{proof}
		      The most trival example is a non empty set $A$ and the empty set $R$.
		      It is vacuously true that $R$ is symmetric and transitive.
		      However, $R$ is not reflexive because there are no relations in it,
		      i.e. the following statement does not hold: $\forall a \in A : (a,a) \in R$.
	      \end{proof}
	      }
\end{enumerate}

\section{}
Give an example of a relation $R$ on $\mathbb{N}$ such that if $S$ is a relation on $\mathbb{N}$
which contains $R$, then $S$ is not antisymmetric. Justify your answer.
(This implies that the antisymmetric closure of R does not exist.)
\begin{proof}
	Consider the relation $R = \{(1,2),(2,1)\}$. It is clear that $1 \neq 2$, hence $R$ is
	not antisymmetric. Therefore, if $S$ contains $R$, then $S$ is not antisymmetric.
\end{proof}

\section{}
Come up with a topological ordering of the following DAG, i.e., write
down the vertices in some order such that if $(i, j)$ is an edge, then $i$ is
to the left of $j$.
\begin{proof}
	Using Kahn's algorithm, we get the topological ordering of
	$2,3,4,6,1,5$.
\end{proof}

\section{}
Let $R$ be the relation on $\mathbb{R}$ defined by $(x, y) \in R$ if and only if $x - y$
is rational. Is $R$ symmetric? Antisymmetric? Reflexive? Transitive?
Justify your answers. (You may use basic properties about rational
numbers without proving them)
\begin{proof}\
	\begin{enumerate}[label=\alph*.]
		\item{
		      R is symmetric\par
		      This is because there is closure under addition/subtraction for rational numbers.
		      If $x - y$ is a rational number then $y - x$ is a rational number, hence
		      $(y,x) \in R$.
		      }
		\item{
		      R is not antisymmetric \par
		      Consider the following relation in $R$: $(1,2)$ and $(2,1)$.
		      It is clear that $1 \neq 2$. Therefore, $R$ is not antisymmetric.
		      }
		\item{
		      R is reflexive \par
		      Suppose $x \in \mathbb{R}$. Any number subtracted by itself will be zero.
		      Zero is a rational number. Thus, $(x,x) \in R$. Therefore, R is reflexive.
		      }
		\item{
		      R is transitive \par
		      Suppose $(x,y)$ and $(y,z)$ are in R. This means that $x - y$ and $y - z$ are rational
		      numbers. Now consider,
		      \begin{align*}
			      x - z & = (x - y) + (y - z)
		      \end{align*}
		      Since there is closure under addition for rational numbers, $x - z$ must be a rational number.
		      Hence, $(x,z) \in R$. Therefore, R is transitive.
		      }
	\end{enumerate}
\end{proof}

%5
\section{}
Let $R$ be a relation on a set $A$. Prove that if $R$ is reflexive, then $R \subseteq R^n$
for all $n \geq 1$.
\begin{proof}
	We proceed by induction on  $n \geq 1$. Let $P(n)$ be the predicate
	if $R$ is reflexive, then $R \subseteq R^n$.\\
	\textbf{Base case:} We prove that $P(1)$ holds. Since $R^1 = R$, $P(1)$ holds.\\
	\textbf{Inductive step:} Suppose that $n \geq 1$ such that $P(n)$ holds. We prove that $P(n + 1)$ holds.
	%Suppose $(a,c) \in R^{n}$. Since $R^{n} = R^{n-1} \circ R$, there must be some some $b \in A$ such that
	%$(a,b) \in R^{n-1}$ and $(b,c) \in R$. 
	Suppose $(a,c) \in R$.
	By our inductive hypothesis, $(a,c) \in R^{n}$.
	Since $R^{n} = R^{n-1} \circ R$, there must be some some $b \in A$ such that
	$(a,b) \in R$ and $(b,c) \in R^{n-1}$.
	Since $R$ is reflexive, there exists $(c,c) \in R$. This directly implies that
	$(a,c) \in R^{n+1}$.
	%We know that $R \subseteq R^n$ by our inductive hypothesis. This means that 
	Therefore, $P(n+1)$ holds and this concludes our inductive step.\\
	We have proved by induction that if $R$ is reflexive, then $R \subseteq R^n$
	for all $n \geq 1$.
\end{proof}

\section{}
Let $R$ be a relation on a set $A$. Prove that for any $m, n \geq 1$, 
$R^n \circ R^m = R^{n+m}$. (You may assume that composition of relations is associative,
i.e., $(Q \circ R) \circ S = Q \circ (R \circ S)$
\begin{proof}
	Fix a positive nonzero $m$. We prove by induction on $n\in \mathbb{N}^+$
	that $R^n \circ R^m = R^{n+m}$.	\\
	\textbf{Base case:} We prove that $P(1)$ holds. Since $n=1$, $R^1 \circ R^m = R^{1+m}$, $P(1)$ holds.\\
	\textbf{Inductive step:} Suppose that $n \in \mathbb{N}^+$ such that $P(n)$ holds. We prove that $P(n + 1)$ holds.
	Since $P(n)$ holds, $R^n \circ R^m = R^{n+m}$. We apply $R$ to both sides,
	\begin{align*}
		R^n \circ R^m & = R^{n+m}\\
		R \circ R^n \circ R^m & = R \circ R^{n+m}\\
		R^{n+1} \circ R^m & = R^{n+m +1}\\
		R^{n+1} \circ R^m & = R^{(n+ 1) +m}
	\end{align*}
	as desired. Therefore, $P(n+1)$ holds and this concludes our inductive step.\\
	We have proved by induction that for any $m, n \geq 1$, 
	$R^n \circ R^m = R^{n+m}$.
\end{proof}

\begin{comment}

Prove that in any group of $n$ people $(n \geq 2),$ there must be two people with the same number of friends within the group. Assume friendship
is mutual, and nobody is their own friend.
\begin{proof}
	Given that there is $n$ people, the amount of friends each person will be in this set: $\{0,1,\cdots, n-1\}$, excluding $n$
	because you cannot be friends with yourself. Consider the case when someone has $n-1$ friends. This means everyone in this group
	has at least one friend. Also consider the case when someone has $0$ friends. This means that the most amount of friends one can
	have is $n-2$ because you cannot be friends with yourself and with the person with $0$ friends. Thus the range of friends that
	a group has is actually $\{1,2,\cdots, n-1\}$ or $\{0,1,\cdots, n-2\}$. In both cases, the number possible combination of friends
	one can have is $n-1$. Using the pigeonhole principle, it directly follows that there must exist at least 2 people with the same
	number of friends in this group.
\end{proof}
\begin{enumerate}[label=\alph*.]
	\item{
	      Let $T$ be a tree. Prove that if you remove a leaf $l$ and the only edge incident with $l$ from $T$,
	      the resulting subgraph $T^{\prime}$ is still a tree.
	      \begin{proof}
		      Since $T$ is a tree, it is by definition acyclic and connected. We must show that $T'$
		      holds the property of a	tree.
		      \begin{enumerate}[label=\roman*.]
			      \item{
			            $T'$ is connected \par
			            From lecture, we know that a graph is connected if and only if there is a simple path between
			            every two distinct vertices. If we removed a leaf and only its edge incident, we know that
			            this cannot affect any simple path from $\{u,v\}$ in the tree. Thus, the resulting tree, $T'$,
			            is connected.
			            }
			      \item{
			            $T'$ is acyclic \par
			            If $T$ is acylic, removing a vertex will not make it cyclic.
			            }
		      \end{enumerate}
		      Therfore, $T'$ is a tree.
	      \end{proof}
	      }
	\item{
	      Prove by induction on $n \geq 1$ that if a tree $T$ has $n$ vertices, then
	      it has exactly $n-1$ edges.
	      \begin{proof}
		      We proceed by induction on  $n \geq 1$. Let $P(n)$ be the predicate
		      "if a tree $T$ has $n$ vertices, then it has exactly $n-1$ edges".\\
		      \textbf{Base case:} We prove that $P(1)$ holds. This means that $T$
		      has $1$ vertex. It trivially follows that $T$ has $0$, i.e. $n-1 = 1 - 1 $ edges. \\
		      \textbf{Inductive step:} Suppose that $n \geq 1$ such that $P(n)$ holds. We prove that $P(n + 1)$ holds.
		      Let $T = (V,E)$ be the tree with $n$ vertices. We know that the tree with $n+1$ edges, $T'$,
		      will have a set of vertices $V' = V \cup \{u\}$, where $u$ is a vertex. Since $T'$ is a tree,
		      $u$ adds only one edge. If it is placed as a leaf, then $u$ there is a new edge create from $u$ to its
		      parent. If it is placed in the middle of the tree, the existing edge between the adjacent vertex will
		      be used as one of the edges connecting $u$ and another edge must be added to finish connecting $u$ to the tree.
		      So in either case, adding a new vertex only adds one edge. It follows then by this fact and by our induction hypothesis that
		      \begin{align*}
			      \sum_{v \in V'} \operatorname{deg}(v) & = \sum_{v \in V} \operatorname{deg}(v) + \operatorname{deg}(u) \\
			                                            & = n - 1 + \operatorname{deg}(u)                                \\
			                                            & = n - 1 + 1                                                    \\
			                                            & = n
		      \end{align*}
		      Therefore, $P(n+1)$ holds and this concludes our inductive step.
		      We have proved by induction that for all $n \geq 1$ that if a tree $T$ has $n$ vertices, then
		      it has exactly $n-1$ edges.
	      \end{proof}
	      }
	\item{
	      Prove that if $G$ is a connected graph with $n$ vertices, then it has at least $n-1$ edges. (Hint: Spanning tree.)
	      \begin{proof}
		      We know that a graph is connected if and only if it has a spanning tree (theorem given in the lecture).
		      Thus, that $G$ is a spanning tree. It directly follows from this fact and from part (b) that $G$ has exactly $n-1$
		      edges.
	      \end{proof}
	      }
\end{enumerate}
\end{comment}

\end{document}
