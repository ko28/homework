\documentclass[11pt]{scrartcl}
\usepackage[sexy]{../../evan}
\usepackage{cmbright}
\usepackage{cancel}
\usepackage[T1]{fontenc}
%\usepackage{enumerate}
\usepackage[shortlabels]{enumitem}
\usepackage[utf8]{inputenc}
\usepackage[margin=1in]{geometry}
%\usepackage[pdfborder={0 0 0},colorlinks=true,citecolor=red{}]{hyperref}
\usepackage{amsmath}
\usepackage{amssymb}
\usepackage{setspace}
\usepackage{systeme}

\makeatletter
\newenvironment{Dequation}
  {%
  \def\tagform@##1{%
    \maketag@@@{\makebox[0pt][r]{(\ignorespaces##1\unskip\@@italiccorr)}}}%
  \ignorespaces
  }
  {%
  \def\tagform@##1{\maketag@@@{(\ignorespaces##1\unskip\@@italiccorr)}}%
  \ignorespacesafterend
  }
\makeatother

\title{Math 341: Homework 6}
\author{Daniel Ko}
\date{Spring 2020}

\begin{document}
\maketitle

\section{A}
Find the rank of the following matrices.

\begin{enumerate}[label=\alph*.]
	\item{
			2
			show work later
	}
	\item{
			3
		}
	\item{
			2
	}
	\item{
			1
		}
	\item{
			3
		}
	\item{
			3
	}
	\item{
			1
		}
\end{enumerate}

\section{B}
Prove that any elementary row [column] operation of type 1 can be obtained by a succession of three elementary
row [column] operations of type 3 followed by one elementary row [column] operation of type 2
\begin{proof}
Row operation type 1 on row $i$ and row $j$ can be done by the following:
\begin{enumerate}[label=\arabic*.]
	\item{
		Row operation type 3: Add $-1$ times row $i$ to row $j$ 
	}
	\item{
		Row operation type 3: Add row $j$ to row $i$
	}
	\item{
		Row operation type 3: Add $-1$ times row $i$ to row $j$
	}
	\item{
		Row operation type 2: Multiply row $j$ by $-1$
	}
\end{enumerate}
Without loss of generality, same could be done for a elementary column operation of type 1.
\end{proof}

\section{C}
Let A be an $mxn$ matrix. Prove that there exists a sequence of elementary row operations of types 1 and 3 that
transforms A into an upper triangular matrix.
\begin{proof}
	Iterate through each column, let this variable be c. If $A_{c,c}$ equals 0, go through all the elements in that column below
	$A_{c,c}$ and find the first non zero element. Perform a type 1 row operation on row $c$ and the row the non zero element
	was found. If there is no non zero element, do nothing and go to the next column. \\
	\-\\
	If $A_{c,c}$ does not equal 0, perform a type 3 row operation on each row below $A_{c,c}$. Multiply 
	$- \frac{A_{r,c}}{A_{c,c}}$ by the $c$th row to the $r$th row, where $r$ is every row below $c$.  
\end{proof}

\section{D}
Complete the proof of the corollary to Theorem 3.4 by showing that elementary column operations preserve rank.
\begin{proof}
	If B is obtained from a matrix A by an elementary column operation,
	then there exists an elementary matrix E such that B = AE. By Theorem 3.2
	(p. 150), E is invertible, and hence rank(B) = rank(A) by Theorem 3.4. 
\end{proof}
	

\section{E}
Let $B$ and $B'$ is an $mxn$ matrix submatrix of B. Prove that rank($B$) = $r$, then rank($B'$) = $r - 1$
\begin{proof}
By Theorem 3.5, we know that rank($B$) = dim($R(L_B)$), where $R(L_B) = \text{span}(B_1,B_2,\cdots,B_{n+1})$
and $B_i$ is the $i$th column of $B$. In other words, is the rank is the number of linearly independent
rows/columns in a matrix. Notice that $B'$ has one less linear independent row/column than $B$. It follows
that rank of $B'$ would be 1 less than the rank of $B$. Therefore, rank($B'$)$= r - 1$.
Consider the matrix 
\[
M=
\left[\begin{array}{@{}c|ccc@{}}
0 & & & \\
\vdots & & B' & \\
0 & & &
\end{array}\right]	
\]
$M$ has the same number of linearly independent columns to that of $B'$, so rank($M$) = rank($B'$)
Now consider the matrix below.
\[
B=
\left[\begin{array}{@{}c|ccc@{}}
1 & 0 &\cdots &0 \\ \hline
\vdots & & B' & \\
0 & & &
\end{array}\right]	
\]
$B$ has one more linearly independent row to that of M.\\ Then rank($B$) = rank($M$) + 1 = rank($B'$) + 1.\\
Therefore, if rank($B$) = r, then rank($M$) = rank($B'$) = r - 1.
\end{proof}
	

\section{F}
Let $B'$ and $D'$ be $mxn$ matrices, and let $B$ and $D$ be $(m + 1)x(n + 1)$ matrices respectively. 
Prove that if $B'$ can be transformed into $D'$ by an elementary row [column] operation, then $B$ can be transformed
into $D$ by an elementary row [column] operation.
\begin{proof}
If $B'$ can be tranformed into $D'$ by elementary row operations, there must exist an elementary matrix $E$ 
such that $D'=EB'$ by theorem 3.1. Now consider the matrix below.
\[
A=
\left[\begin{array}{@{}c|ccc@{}}
1 & 0 &\cdots &0 \\ \hline
\vdots & & E & \\
0 & & &
\end{array}\right]	
\]
A is also an elementary matrix, 
We can observe that $D = AB$, thus $B$ can be tranformed to $D$ by elementary row operations.
Without loss of generality, there exist a matrix $F$ such that $D' = B'F$ where $F$ is the elementary column matrix.
So, $D = BF$ where $B$ is like the $A$ matrix but with $F$ instead of $E$. Therefore. $B$ can be tranformed
to $D$ by elementary column operations. 
\end{proof}

\section{G}

\begin{enumerate}[label=\alph*.]
	\item{
		Find a $5x5$ matrix $M$ with rank 2 such that $AM=O$ where $O$ is the $4x5$ zero matrix.
		\begin{proof}
			By solving $Ax=0$, we get this system of equation:\\
			\systeme{
				x_1 - x_3 +  2x_4 + x_5 = 0,
				-x_1 + x_2 + 3x_3 - x_4 = 0,
				-2x_1 + x_2 + 4x_3 - x_4 + 3x_5 = 0,
				3x_1 - x_2 - 5x_3 + x_4 - 6x_5 = 0}\\
			Solving this system of equations by computing reduced row echelon form, 
			we get that $x_1, x_2, x_4$ are the pivot variables and $x_3, x_5$ are the 
			free variables. So solutions are in the form 
			$(x_3+3x_5,-2x_3+x_5, x_3, -2x_5,x_5)$. From this, we are able to 
			construct a basis for Ax=0, $(1,-2,1,0,0),(3,1,0,-2,1)$.
			Define   
			\[			
				M=
				\left[\begin{array}{ccccc}
				1 &3 & 0 & 0 &0 \\ 
				-2 &1 & 0 & 0 &0 \\ 
				1 &0 & 0 & 0 &0 \\ 
				0 &-2 & 0 & 0 &0 \\ 
				0 &1 & 0 & 0 &0 
				\end{array}\right]	
			\]
			It is obvious that this matrix has rank 2 and is $5x5$. Because the column is a basis for $Ax=0$, the resulting matrix will be $O$.
		\end{proof}
	}
	\item{
		Suppose that $B$ is a $5x5$ matrix such that $AB=O$. Prove that rank$(B) \leq 2$
		\begin{proof}
			Since $AB=O$, we know that the columns of $B$ is a solution to $Ax=0$, which is a subset of the nullspace of $L_A$.
			From the rank nullity theorem, we know that $\text{dim}(\mathbb{F}^5) = \text{rank}(L_A) + \text{nullity}(L_A)$.\\
			$\text{nullity}(L_A) = \text{dim}(\mathbb{F}^5) - \text{rank}(L_A) = 5 - 3 = 2$. So, rank($B$) cannot be greater than 2. 
			Therefore, rank$(B) \leq 2$.
		\end{proof}
	}
\end{enumerate}

\section{H} 
For each of the following linear transformations $T$, determine whether $T$ is invertible, and compute $T^{-1}$ if it exists.
\begin{enumerate}[label=\alph*.]
	\item{
	$T : P_2(R) \rightarrow P_2(R)$ defined by $T(f(x)) = f''(x) + 2f'(x) - f(x)$
	\begin{proof}
	\-\\
	We want to projection to be on the xy-plane along the z-axis. Let the projection be (x,y,0).\\
	To minimize the distance, we must choose x and y such that
	$$(a - x)^2 + (b - y)^2 + (c - 0)^2$$
	is minimum.
	Since the equation above is a difference of squares, x = a and b = y will give us the minimum value.
	Therefore, the projection on the xy-plane will be (a,b,0), which is T.
	\end{proof}
	}
	
	\item{
	Find a formula for T(a,b,c), where T represents the projection on the z-axis along the xy-plane.
	\begin{proof}
	\-\\
	We want to projection to be on the z-axis along the xy-plane. Let the projection be (0,0,z).\\
	To minimize the distance, we must choose z such that
	$$(a - 0)^2 + (b - 0)^2 + (c - z)^2$$
	is minimum.
	z = c will give us the minimum value.
	Therefore, the equation for T will be T(a,b,c)=(0,0,c).
	\end{proof}
	}

	\item{
	If T(a,b,c) = (a-c,b,0), show that T is the projection on the xy-plane along the line L $= \{(a,0,a):a \in R\}$
	\begin{proof}
	\-\\
	We want to projection to be on the xy-plane along the line L. Let the projection be $(x,y,0)$.\\
	A vector that is on L is $(1,0,1)$.
	To minimize the distance, we must choose $\lambda$ such that
	$$(a,b,c) + \lambda (1,0,1) = (x,y,0)$$
	is minimum. Writing the equation above as a system:
	\[
	\begin{aligned}
		a + \lambda &= x\\
		b &= y\\
		c + \lambda &= 0
	\end{aligned}
	\]
	Solving this system gives us, $x = a - c, y = b$\\ 
	Therefore, the projection on the xy-plane along the line L will be $(a-c,b,0)$.\\
	\end{proof}
	}

\end{enumerate}



\section{I} 
Express the invertible matrix
$
\begin{bmatrix}
1 & 2 & 1\\
1 & 0 & 1 \\ 
1 & 1 & 2 \\ 
\end{bmatrix}
$
as a product of elementary matrices
\begin{proof}

\end{proof}

\section{J}
Suppose that $A$ and $B$ are matrices having $n$ rows. Prove that $M(A|B) = (MA|MB)$ for any $mxn$ matrix
\end{document}