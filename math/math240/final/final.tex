\documentclass[11pt]{scrartcl}
\usepackage[sexy]{../../evan}
\usepackage{cmbright}
\usepackage{cancel}
\usepackage[T1]{fontenc}
%\usepackage{enumerate}
\usepackage[shortlabels]{enumitem}
\usepackage[utf8]{inputenc}
\usepackage[margin=1in]{geometry}
%\usepackage[pdfborder={0 0 0},colorlinks=true,citecolor=red{}]{hyperref}
\usepackage{amsmath}
\usepackage{amssymb}
\usepackage{setspace}
\usepackage{systeme}

\makeatletter
\newenvironment{Dequation}
{%
	\def\tagform@##1{%
	\maketag@@@{\makebox[0pt][r]{(\ignorespaces##1\unskip\@@italiccorr)}}}%
	\ignorespaces
}
{%
	\def\tagform@##1{\maketag@@@{(\ignorespaces##1\unskip\@@italiccorr)}}%
	\ignorespacesafterend
}
\makeatother

\title{Math 240: Final Written}
\author{Daniel Ko}
\date{Spring 2020}

\begin{document}
\maketitle

\section{}
Give an example of a tree $T$ and a vertex $v \in T$ such that the induced subgraph $T'$ of $T$ with vertex set
$V-\{v\}$ is not connected. (In other words, removing $v$ from $T$ makes it disconnected.)
To receive full credit, you must describe $T, v$ and $T^{\prime}$ explicitly (pictures are sufficient, but you must draw both $T$ and $T^{\prime}$ ), and explain why $T^{\prime}$ is not connected
\begin{proof}
	Consider a tree $T$ with $V = \{v_1,v_2,v_3\}$ and $E = \{(v_1,v_2),(v_2,v_3)\}$. It is clear that $T$ is connected,
	i.e. there is a path between every two distinct vertices in the graph. Define the induced subgraph $T'$ of $T$
	with vertex set $V-\{v_2\}$. Notice that there will be no edges in $T'$ because in $T$ all the edges had $v_2$ as
	one of its endpoints. Hence $V' =\{v_1,v_3\}$ and $E' = \{\}$. $T'$ is not connected because there does not
	exist a path between every two distinct vertices in the graph, namely there is no path between $v_1$ and $v_3$.
\end{proof}

\section{}
For this question, you may leave your answers as formulas, rather than calculating the exact numerical answers.
For full credit, show all your work and explain your steps.
\begin{enumerate}[label=\alph*.]
	\item {
	      How many binary strings of length seven contain at least three 0’s?
	      \begin{proof}
		      We want to count the possible number of ways having three 0's in length seven string,
		      four 0's in length seven string, $\cdots$, seven 0's in length seven string and adding it all up.
		      $$
			      \sum_{i=3}^{7} C(7,i)
		      $$
	      \end{proof}
	      }
	\item{
	      How many binary strings of length seven contain at least four 1’s?
	      \begin{proof}
		      We want to count the possible number of ways having four 1's in length seven string,
		      five 1's in length seven string, $\cdots$, seven 1's in length seven string and adding it all up.
		      $$
			      \sum_{i=4}^{7} C(7,i)
		      $$
	      \end{proof}
	      }
	\item{
	      How many binary strings of length seven contain at least three 0’s \textbf{and} at least four 1’s?
	      \begin{proof}
		      Since the string is length seven, the above question is equivalent of counting the number of
		      strings of length 7 that have three 0's and four 1's. We can just count either 0's or 1's
		      because once we fix the 0's the rest must be 1's and vice versa.
		      %Since the ones and zeros are not unique (i.e. 101 and 101 are the same) we can compute 

		      $$
			      C(7,3) = C(7,4)
		      $$
	      \end{proof}
	      }
	\item{
	      How many binary strings of length seven contain at least three 0’s \textbf{or} at least four 1’s?
	      \begin{proof}
		      Using the inclusion exclusion principle, we add the number of strings with least three 0’s
		      and the number of strings with least four 1’s, and the we subtract the number of strings with
		      at least three 0's \textbf{and} least four 1’s.
		      $$
			      \left( \sum_{i=3}^{7} C(7,i) \right) + \left(  \sum_{i=4}^{7} C(7,i) \right) - C(7,3)
		      $$
	      \end{proof}
	      }
\end{enumerate}

\section{}
Let $f: A \rightarrow B$ be a function from a set $A$ to a set $B$. Prove that
$R_{f} \circ R_{f}^{-1} \subseteq\{(b, b): b \in B\}$.

\begin{proof}
	Let $x = (\alpha,\theta) \in R_{f} \circ R_{f}^{-1}$. This means that there exists a $\phi$ such that
	$(\alpha,\phi) \in R_{f}^{-1}$ and $(\phi,\theta) \in R_{f}$. We also know that $\alpha,\theta \in B$ and $\phi \in A$.
	Observe that since $f$ is an isomorphism,
	\begin{align*}
		f^{-1} (\alpha)     & = \phi    \\
		f (f^{-1} (\alpha)) & = f(\phi) \\
		\alpha              & = f(\phi)
	\end{align*}
	This directly implies that $\alpha = \theta$ because $(\phi,\theta) \in R_{f}$. Hence, $x \in \{(b, b): b \in B\}$.\\
	Therefore, $R_{f} \circ R_{f}^{-1} \subseteq\{(b, b): b \in B\}$.
\end{proof}

\section{}
Suppose that a graph $G$ satisfies the following properties:
\begin{itemize}
	\item $G$ has $n$ vertices
	\item $G$ has $m$ edges, where $m<n$
	\item Every vertex in $G$ has degree at least 1
\end{itemize}
Prove that there are at least $2 (n-m)$ vertices which have degree 1.
\begin{proof}
	By the handshake lemma, $\sum_{v \in V} \text{deg}(v) = 2m$.
	Let $\alpha$ be the number of vertices with degree 1. Then $n - \alpha$ will be the
	number of vertices with degree greater than 1. Then we can construct the following inequality
	where the rest of the vertices will have at least degree 2, 
	\begin{align*}
		2m      & \geq \alpha +  2(n - \alpha) \\
		2m - 2n & \geq - \alpha             \\
		2(n-m) & \leq \alpha             \\
		% \sum_{v \in V : deg(v) = 1} \text{deg}(v) +  \sum_{v \in V : deg(v) > 1} \text{deg}(v) \\  
		%2m -  \sum_{v \in V : deg(v) > 1} \text{deg}(v) & =  \sum_{v \in V : deg(v) = 1} \text{deg}(v)\\
	\end{align*}
	as desired.

\end{proof}


\end{document}
