\documentclass[11pt]{scrartcl}
\usepackage[sexy]{../../evan}
\usepackage{cmbright}
\usepackage{cancel}
\usepackage[T1]{fontenc}
%\usepackage{enumerate}
\usepackage[shortlabels]{enumitem}
\usepackage[utf8]{inputenc}
\usepackage[margin=1in]{geometry}
%\usepackage[pdfborder={0 0 0},colorlinks=true,citecolor=red{}]{hyperref}
\usepackage{amsmath}
\usepackage{amssymb}
\usepackage{setspace}

\makeatletter
\newenvironment{Dequation}
  {%
  \def\tagform@##1{%
    \maketag@@@{\makebox[0pt][r]{(\ignorespaces##1\unskip\@@italiccorr)}}}%
  \ignorespaces
  }
  {%
  \def\tagform@##1{\maketag@@@{(\ignorespaces##1\unskip\@@italiccorr)}}%
  \ignorespacesafterend
  }
\makeatother

\title{Math 341: Homework 2}
\author{Daniel Ko}
\date{Spring 2020}

\begin{document}
\maketitle

\section{A}
Let $D$ be the set of all differentiable functions defined on $\mathbb{R}$. Note that $D$ is a subset of $C$ because differentiable functions are continuous.

\begin{proof}
$D$ is a subspace of $C$
\begin{enumerate}[label=\alph*.]
	\item{
			$0 \in D$\\
			$\text{Zero vector is defined as } f(x) = 0 \text{ where } x \in \mathbb{R} $
			\begin{Dequation}
			\begin{align*}
				f'(x) & = \lim_{h \to 0} \frac{f(x+h) - f(x)}{h}\\
					  & = \lim_{h \to 0} \frac{0 - 0}{0} \\
					  & = 0
			\end{align*}
			\end{Dequation}
			Because the derivative of $f(x) = 0$ exists, $0 \in D$
		}
	\item{
			$f + g \in D \text{ where } f,g \in D $
			\begin{Dequation}
			\begin{align*}
				(f + g)'(x) & = \lim_{h \to 0} \frac{(f(x+h) + g(x+h)) - (f(x)+g(x))}{h}\\
							& = \lim_{h \to 0} \frac{f(x+h) - f(x)}{h} + \lim_{h \to 0} \frac{g(x+h)-g(x)}{h}\\
					  & = f'(x) + g'(x)
			\end{align*}
			\end{Dequation}
			Because the derivative of $f + g$ exists, $f + g \in D$
		}
	\item{
			$cf \in D \text { where } c \in \mathbb{R} \text{ and } f \in D$
			\begin{Dequation}
			\begin{align*}
				cf'(x) & = \lim_{h \to 0} \frac{cf(x+h) - cf(x)}{h}\\
				& = c \lim_{h \to 0} \frac{f(x+h) - f(x)}{h}
			\end{align*}
			\end{Dequation}
			Because the derivative of $cf$ exists, $cf \in D$
		}
\end{enumerate}
\begin{center}
$\therefore D \text{ is a subspace of } C$
\end{center}
\end{proof}

\section{B}
Prove the set of even functions in $F(F_1, F_2)$ and odd functions in $F(F_1, F_2)$ are subspaces of $F(F_1, F_2)$ 
\begin{proof}
Let $O$ be the set of all odd functions in $F(F_1, F_2)$ and $E$ be the set of all even functions in $F(F_1, F_2)$.
\begin{enumerate}[label=\alph*.]
	\item{
			$0 \in O$ and $0 \in E$\\
			$\text{Zero function is defined as } g(x) = 0$\\
			\\$0\in O$ is odd: 
			\begin{Dequation}
			\begin{align*}
				g(-x) & = 0\\
				-g(x) & = 0\\
				g(-x) & = -g(x)
			\end{align*}
			\end{Dequation}

			\-\\$0\in E$ is even: 
			\begin{Dequation}
			\begin{align*}
				g(x) & = 0\\
				g(-x) & = 0\\
				g(x) & = g(-x)
			\end{align*}
			\end{Dequation}
		}
	
	\item{
		$X + Y \in O \text{ where } X,Y \in O$ and $t \in F_1$
		\begin{Dequation}
		\begin{align*}
			(X + Y)(-t) & = X(-t) + Y(-t) \\
						& = -X(t) + -Y(t)\tag{$X,Y \in O$}\\
			& = -(X + Y)(t)
		\end{align*}
		\end{Dequation}
		$X + Y \in E \text{ where } X,Y \in E$ and $t \in F_1$
		\begin{Dequation}
		\begin{align*}
			(X + Y)(t) & = X(t) + Y(t) \\
						& = X(-t) + Y(-t)\tag{$X,Y \in E$}\\
			& = (X + Y)(-t)
		\end{align*}
		\end{Dequation}
		}
	\item{
			$cX \in O \text{ where } c \in F \text{ and } X \in O \text{ and } t \in F_1$
			\begin{Dequation}
			\begin{align*}
				(cX)(-t) & = cX(-t) \\
						 & = -cX(t) \\
			\end{align*}
			\end{Dequation}
			
			$cY \in E \text{ where } c \in F \text{ and } Y \in E \text{ and } t \in F_1$
			\begin{Dequation}
			\begin{align*}
				(cY)(t) & = cY(t) \\
						 & = cY(-t) \\
			\end{align*}
			\end{Dequation}
		}
	\end{enumerate}	
	Therefore, $O$ and $E$ are subspaces of $F(F_1,F_1)$
\end{proof}

\section{C}
$W_1 = \{(a_1, a_2, \cdots, a_n) \in F^n : a_n = 0 \}$ \\
$W_2 = \{(a_1, a_2, \cdots, a_n) \in F^n : a_1 = a_2 = \cdots = a_{n-1} = 0 \}$ \\
Show $F^n$ = $W_1 \oplus W_2$
\begin{proof}
Definition of direct sum is $W_1 \cap W_2 = \{0\}$ and $W_1 + W_2 = F^n$
\begin{enumerate}[label=\alph*.]
	\item{
		$W_1 \cap W_2 = \{0\}$\\
		\-\\Let $v \in W_1,W_2$\\
		$v = (a_1, a_2, \cdots, a_n)$\\
		$v \in W_1 \Rightarrow a_n = 0$\\
		$v \in W_2 \Rightarrow a_1 = a_2 = \cdots = a_{n-1} = 0 $\\
		$\therefore v = (0, 0, \cdots, 0) \Rightarrow W_1 \cap W_2 = \{0\}$
		}
	\item{
		$W_1 + W_2 = F^n$\\
		\-\\ Let $v \in F^n$\\
		$v = (a_1, a_2, \cdots, a_n)$\\
		Let $w_1 \in W_1$ and $w_2 \in W_2$\\
		$w_1 = (a_1, a_2, \cdots, a_{n-1}, 0)$\\
		$w_2 = (0, 0, \cdots, a_n)$\\
		$w_1 + w_2 = (a_1, a_2, \cdots, a_n) = v$\\
		Thus, any vector in $F^n$ can be expressed as a sum of vectors in $W_1$ and $W_2$\\
		$\therefore W_1 + W_2 = F^n$
		}
\end{enumerate}
$\therefore F^n = W_1 \oplus W_2$
\end{proof}

\section{D}
In $M_{mxn}(F)$\\
$W_1 = \{A\in M_{mxn}(F): A_{i,j} = 0 \text{ whenever }i>j\}$\\
$W_2 = \{B\in M_{mxn}(F): B_{i,j} = 0 \text{ whenever }i \leq j\}$\\
Show that $M_{mxn}(F) = W_1 \oplus W_2$
\begin{proof}
\-\
\begin{enumerate}[label=\alph*.]
	\item{
			$W_1 \cap W_2 = \{0\}$\\
			\-\\ Let $m \in W_1, W_2$\\
			$m \in W_1 \Rightarrow m_{i,j} = 0 \text{ whenever }i>j$\\
			$m \in W_2 \Rightarrow m_{i,j} = 0 \text{ whenever }i \leq j$\\
			Thus, $(\forall i,j)(m_{i,j} = 0)$ which is \{0\}\\
			$\therefore W_1 \cap W_2 = \{0\}$\\
		}	
	\item{
			$W_1 + W_2 = M_{mxn}(F)$\\
			\-\\ Let $q \in M_{mxn}(F)$\\
			%$q = (a_1, a_2, \cdots, a_n)$\\
			Let $w_1 \in W_1$ and $w_2 \in W_2$\\
			$w_1 = \{(w_1)_{i,j} = 0 \text{ whenever }i>j \}$\\ 
			$w_2 = \{(w_2)_{i,j} = 0 \text{ whenever }i\leq j \}$\\ 
			$w_1 + w_2 = \{(w_1)_{i,j} \text{ wherever } i \leq j \text{ and } (w_2)_{i,j} \text{ wherever } i>j \} = q$\\
			Thus, any matrix in $M_{mxn}(F)$ can be expressed as a sum of matrices in $W_1$ and $W_2$\\
			$\therefore W_1 + W_2 = M_{mxn}(F)$
		}	
\end{enumerate}
$\therefore M_{mxn}(F) = W_1 \oplus W_2$
\end{proof}

\section{E}
Let $W$ be a subspace of a vector space $V$ over a field $F$.\\
For any $v \in V$ the set $\{v\} + W = \{v + w : w \in W \}$ is the coset $W$ containing $v$.
\begin{enumerate}[label=\alph*.]
	\item{
		Prove that $v + W$ is in the subspace of $V$ if and only if $v \in W$.
		\begin{proof}
			\-\\
			$v + W$ is in the subspace of $V \Rightarrow v \in W$.\\
			$0 \in v + W$ because $v + W$ is a subspace.\\
			$0 = v + w, w \in W$\\
			$v = -w$\\
			$v \in W$\\ 
			\-\\
			$v \in W \Rightarrow v + W$ is in the subspace of $V$.
			\begin{enumerate}[i.]
			\item{
			$0 \in v + W$\\
			$w \in W$ and let $v = -w$\\
			$v + w = 0$\\
			Thus, $0 \in v + W$
			}
			\item{
			$a + b \in v + W$ where $a,b \in v + W$\\
			Let $a = v + w_a, w_a \in W$ and $b = v + w_b, w_b \in W$ \\
			$a + b = v + w_a + v + w_b$\\
			Because $v \in W$, $w_a + v + w_b \in W$.\\
			Thus, $a + b \in v + W$
			}
		\item{
			$ca \in v + w, a \in v+W, c \in F$\\
			Let $a = v + w_a, w_a \in W$\\
			$ca = c(v + w_a)$\\
			$= cv + cw_a$\\
			$= v + cv + cw_a - v$\\
			$cv + c_wa - v \in W$ by closure under scalar multplication and vector addition.\\
			Thus, $ca \in v + w$
			}
		\end{enumerate}
		\end{proof}
	\item{
		 Prove that $v_1 + W = v_2 + W$ if and only if $v_1 - v_2 \in W$
		 \begin{proof}
			 \-\
			\begin{enumerate}[i.]
			\item{
				$v_1 + W = v_2 + W \Rightarrow v_1 - v_2 \in W$\\
				Let $w_1, w_2 \in W$\\
				$v_1 + w_1 = v_2 + w_2$\\
				$v_1 - v_2 = w_2 - w_1$\\
				Since, $w_2 - w_1 \in W$ (clourse under addition)\\
				Therefore, $v_1 - v_2 \in W$
				}
			\item{
				$v_1 - v_2 \in W \Rightarrow v_1 + W = v_2 + W $\\
				This means $v_1 - v_2 = w$ where $w \in W \ (*)$\\
				Now let $x \in v_1 + W$\\
				By definition, $\exists w_x \in W : x = v_1 + w_x$\\
				By $(*) \ v_1 = v_2 + w$\\
				So, $x = v_2 + w + w_x$\\
				Since, $w + w_x \in W$ (closure under addition)\\
				We have $x \in v_2 + W$\\
				So, $v_1 + W \subseteq v_2 + W$\\
				Similarly, we can show $v_2 + W \subseteq v_1 + W$\\
				Therefore, $v_1 + W = v_2 + W$

				}
			\end{enumerate}
		 \end{proof}
		}
	}	
\end{enumerate}

\section{F}
Show that if 
\[
M_1 = 
\begin{pmatrix}
1 & 0 \\
0 & 0 
\end{pmatrix},
M_2 = 
\begin{pmatrix}
0 & 0 \\
0 & 1 
\end{pmatrix},
\
M_3 = 
\begin{pmatrix}
0 & 1 \\
1 & 0 
\end{pmatrix}
\]
then the span of $\{M_1, M_2, M_3\}$ is the set of all symmetric $2x2$ matrices.

\begin{proof}
	\-\\
	$Sym(M_{2x2}(F)) = \{m \in M_{2x2}(F): m = \begin{pmatrix} a & b \\ b & c \end{pmatrix}  \Leftrightarrow m = m^t\}$
	\begin{Dequation}
	\begin{align*}
		m \in span(\{M_1, M_2, M_3\}) \text{ if } m & = c_1 
	\begin{pmatrix}
	1 & 0 \\
	0 & 0 
	\end{pmatrix}+
	c_2
	\begin{pmatrix}
	0 & 0 \\
	0 & 1 
	\end{pmatrix}+
	c_3
	\begin{pmatrix}
	0 & 1 \\
	1 & 0 
	\end{pmatrix}, \text{where } c_1, c_2, c_3 \in F\\
	& =\begin{pmatrix}
	c_1 & c_3 \\
	c_3 & c_2 
	\end{pmatrix}\\
		m^t & = 
	\begin{pmatrix}
	c_1 & c_3 \\
	c_3 & c_2 
	\end{pmatrix}
	\end{align*}
	\end{Dequation}
	$\therefore Sym(M_{2x2}(F)) = span(\{M_1, M_2, M_3\})$
\end{proof}



\end{document}
