\documentclass[11pt]{scrartcl}
\usepackage[sexy]{../../../evan}
\usepackage{cmbright}
\usepackage{cancel}
\usepackage[T1]{fontenc}
%\usepackage{enumerate}
\usepackage[shortlabels]{enumitem}
\usepackage[utf8]{inputenc}
\usepackage[margin=1in]{geometry}
%\usepackage[pdfborder={0 0 0},colorlinks=true,citecolor=red{}]{hyperref}
\usepackage{amsmath}
\usepackage{amssymb}
\usepackage{setspace}
\usepackage{systeme}

\makeatletter
\newenvironment{Dequation}
{%
	\def\tagform@##1{%
	\maketag@@@{\makebox[0pt][r]{(\ignorespaces##1\unskip\@@italiccorr)}}}%
	\ignorespaces
}
{%
	\def\tagform@##1{\maketag@@@{(\ignorespaces##1\unskip\@@italiccorr)}}%
	\ignorespacesafterend
}
\makeatother

\title{Math 240: Midterm Q1, Q2, Q3}
\author{Daniel Ko}
\date{Spring 2020}

\begin{document}
\maketitle

\section{Q1}
Solve the following recurrence relations. Show your work. If you wish,
you may use the formulas we stated in lecture, without proof
\begin{enumerate}[label=\alph*.]
	\item{
		$ a_{1}=1, a_{n+1}=3 a_{n}+6 \text { for } n \geq 1$
		\begin{proof}
			This is a linear homogeneous recurrence relation of degree one.
			Given any recurrence relations in this form $a_n = ra_{n-1} + c$, we can rewrite the relation as
			$a_{n}=r^{n} a_{0}+c \frac{r^{n}-1}{r-1}$ if $r \neq 1$. We convert the above relation into 			
			\[
				a_{n}=3 a_{n-1}+6	
			\]
			We must replace $n$ on the right side of the formula with $n-1$ since we are starting
			at $a_1$.
			We solve for the recurrence relation
			\begin{align*}
				a_{n} & = 3^{n-1}(1) + 6 \frac{3^{n-1}-1}{3-1} \\
				& = 3^{n-1} + 3(3^{n-1}-1)\\
				& = 4(3^{n-1}) - 3
			\end{align*}
			for $n \geq 1$ as desired.
		\end{proof}
	}
	\item{
		$a_{0}=1, a_{1}=-3, a_{n}=a_{n-1}+12 a_{n-2}$ for $n \geq 2$
		\begin{proof}
			Notice that this is a linear homogeneous recurrence relation of degree two.
			We can write the above recurrence relation as $r^2 - r - 12 = 0$. Solving for $r$ gets us $r = -3, r = 4$.
			We use the equation given in the lecture for linear homogeneous recurrence relations of degree two.
			\begin{align*}
			a_n & = \alpha(-3)^n + \beta(4)^n\\
			a_{0}=1 \Rightarrow 1 & = \alpha + \beta \\
			a_{1}=-3 \Rightarrow -3 & = \alpha (-3) + \beta (4)\\
			\alpha = 1 \quad & \quad \beta = 0\\ 
			a_n & =  (1) (-3)^n + (0)(4)^n \\
			& = (-3)^n
			\end{align*}
			for $n \in \mathbb{N}$ as desired.
		\end{proof}
	}
\end{enumerate}

\section{Q2}
Consider a recursive algorithm SEQ, defined as follows. Given a
positive integer $n$, SEQ does the following:
\begin{verbatim}
Initialize variables x = 0 and y = 0
If n = 1, return 2
If n = 2, return 4
If n >= 3, do the following:
    Make a recursive call to SEQ(n - 1) and put its return value in x
    Make a recursive call to SEQ(n - 2) and put its return value in y
    Return x + y
\end{verbatim}
Formulate a recurrence relation T(n) which expresses the number of addition operations
that the above algorithm performs when given a positive integer n as input.
(You do not have to justify your answer. You do not have to solve the recurrence relation.
For the purposes of this question, we assume that each recursive call to SEQ occurs separately. 
In particular, you may not reuse information from the computation of SEQ(n - 1)
in the computation of SEQ(n - 2), or vice versa.)
\begin{proof}
	\[
		T(n) = T(n-1) + T(n-2) + 1
	\]
	where $T(1) = T(2) = 0$.
\end{proof}


\section{Q3}
Using the master theorem, what is the growth rate of
\[
	T(n) = 7T(n/2) + 3n^2	
\]
in terms of big O-notation? (State the values of the various constants that you plug in to
the master theorem. State which case of the master theorem you are applying.)
\begin{proof}
	We observe that 
	\begin{align*}
		a & = 7\\
		b & = 2\\
		d & = 2
	\end{align*}
	This means that $a > b^d$ because $7 > 2^2$. Using the master theorem, 
	$f(n)$ is $O(n^{\log_2(7)})$
\end{proof}
\end{document}
