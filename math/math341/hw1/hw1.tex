\documentclass[11pt]{scrartcl}
\usepackage[sexy]{../../evan}
\usepackage{cmbright}
\usepackage{cancel}
\usepackage[T1]{fontenc}
%\usepackage{enumerate}
\usepackage[shortlabels]{enumitem}
\usepackage[utf8]{inputenc}
\usepackage[margin=1in]{geometry}
%\usepackage[pdfborder={0 0 0},colorlinks=true,citecolor=red{}]{hyperref}
\usepackage{amsmath}
\usepackage{amssymb}
\usepackage{setspace}

\makeatletter
\newenvironment{Dequation}
  {%
  \def\tagform@##1{%
    \maketag@@@{\makebox[0pt][r]{(\ignorespaces##1\unskip\@@italiccorr)}}}%
  \ignorespaces
  }
  {%
  \def\tagform@##1{\maketag@@@{(\ignorespaces##1\unskip\@@italiccorr)}}%
  \ignorespacesafterend
  }
\makeatother

\title{Math 341: Homework 1}
\author{Daniel Ko}
\date{Spring 2020}

\begin{document}
\maketitle

\section{A}
\begin{enumerate}[label=\alph*.]
	\item{
		$p \Rightarrow (p \lor q)$
		\begin{center}
		\begin{displaymath}
		\begin{array}{|c|c|c|c|}
			\hline
			p & q & p \lor q & p \Rightarrow (p \lor q) \\ 
			\hline
			T & T & T & T\\
			T & F & T & T\\
			F & T & T & T\\
			F & F & F & T\\
			\hline
		\end{array}
		\end{displaymath}
		\end{center}
		}
	\item{
		$p \lor F \Leftrightarrow F$
		\begin{center}
		\begin{displaymath}
		\begin{array}{|c|c|c|c|}
			\hline
			p & F & p \lor F & p \lor F \Leftrightarrow F \\ 
			\hline
			T & F & T & T\\
			F & F & F & T\\
			\hline
		\end{array}
		\end{displaymath}
		\end{center}
		}
	\item{
		$p \land \lnot p \Leftrightarrow F$
		\begin{center}
		\begin{displaymath}
		\begin{array}{|c|c|c|c|}
			\hline
			p & \lnot p & p \land \lnot p & p \land \lnot p \Leftrightarrow F\\ 
			\hline
			T & F & F & T\\
			F & T & F & T\\
			\hline
		\end{array}
		\end{displaymath}
		\end{center}
		}
	\item{
			$(p\Leftrightarrow q) \Leftrightarrow [(p \land q) \lor (\lnot p \land \lnot q)]$
		\begin{center}
		\begin{displaymath}
		\begin{array}{|c|c|c|c|c|c|c|}
			\hline
			p & q & p\Leftrightarrow q & p \land q & \lnot p \land \lnot q & (p \land q) \lor (\lnot p \land \lnot q) & (p\Leftrightarrow q) \Leftrightarrow [(p \land q) \lor (\lnot p \land \lnot q)]\\
			\hline
			T & T & T & T & F & T & T\\
			T & F & F & F & F & F & T\\
			F & T & F & F & F & F & T\\
			F & F & T & F & T & T & T\\
			\hline
		\end{array}
		\end{displaymath}
		\end{center}
		}
	\item{
			$[(p \Leftrightarrow q) \land (q \Leftrightarrow r)] \Rightarrow (p \Leftrightarrow r)$ 
		\begin{center}
		\begin{displaymath}
		\begin{array}{|c|c|c|c|c|c|c|c|}
			\hline
			p & q & r & p \Leftrightarrow q & q \Leftrightarrow r & (p \Leftrightarrow q) \land (q \Leftrightarrow r) & p \Leftrightarrow r & [(p \Leftrightarrow q) \land (q \Leftrightarrow r)] \Rightarrow (p \Leftrightarrow r) \\ 
			\hline
			T & T & T & T & T & T & T & T\\
			T & T & F & T & F & F & F & T\\
			T & F & T & F & F & F & T & T\\
			T & F & F & F & T & F & F & T\\
			F & T & T & F & T & F & T & T\\
			F & T & F & F & F & F & T & T\\
			F & F & T & T & F & F & F & T\\
			F & F & F & T & T & T & T & T\\
			\hline
		\end{array}
		\end{displaymath}
		\end{center}
		}
	\item{
			$[p\land \lnot q \Rightarrow \lnot p] \Rightarrow (p \Rightarrow q)$
		\begin{center}
		\begin{displaymath}
		\begin{array}{|c|c|c|c|c|c|c|}
			\hline
			p & q & p \land \lnot q & (p \land \lnot q) \Rightarrow \lnot p & p \Rightarrow q & [p\land \lnot q \Rightarrow \lnot p] \Rightarrow (p \Rightarrow q) \\
			\hline
			T & T & F & T & T & T\\
			T & F & T & F & F & T\\
			F & T & F & T & T & T\\
			F & F & F & T & T & T\\
			\hline
		\end{array}
		\end{displaymath}
		\end{center}
		}

\end{enumerate}

\section{B}
\begin{enumerate}[label=\alph*.]
	\item{
		$(p \lor q \Leftrightarrow p \land r) \Rightarrow ((p \Rightarrow p) \land (p \Rightarrow r))$\\
		\begin{align*}
			(p \lor q \Leftrightarrow p \land r) & \Rightarrow (p \Rightarrow r) \tag{Transitivity}\\
			[(p \lor q \Rightarrow p \land r) \land (p \land r \Rightarrow p \lor q)] & \Rightarrow (p \Rightarrow r) \tag{Def. of bicondtional}\\
			\lnot [(p \lor q \Rightarrow p \land r) \land (p \land r \Rightarrow p \lor q)] & \lor (\lnot p \lor r) \tag{Material Implication}\\
			\lnot [(\lnot (p \lor q) \lor (p \land r)) \land (\lnot(p \lor r) \lor (p \lor q))] & \lor (\lnot p \lor r) \tag{Material Implication}\\
			[\lnot(\lnot (p \lor q) \lor (p \land r)) \lor \lnot(\lnot(p \lor r) \lor (p \lor q))] & \lor (\lnot p \lor r) \tag{De Morgan's Law}\\
			(\lnot \lnot (p \lor q) \land \lnot (p \land r)) \lor (\lnot \lnot(p \lor r) \land \lnot (p \lor q)) & \lor (\lnot p \lor r) \tag{De Morgan's Law}\\
			((p \lor q) \land \lnot (p \land r)) \lor ((p \lor r) \land \lnot (p \lor q)) & \lor (\lnot p \lor r) \tag{Double negation}\\
			(\lnot p \lor r) \lor ((p \lor q) \land \lnot (p \land r)) \lor ((p \lor r) & \land \lnot (p \lor q)) \tag{Commutative}\\
			((\lnot p \lor r) \lor (p \lor q)) \land ((\lnot p \lor r) \lor \lnot (p \land r)) \lor ((p \lor r) & \land \lnot (p \lor q)) \tag{Distributive} \\
			(True \land True) \lor ((p \lor r) &\land \lnot (p \lor q)) \tag{Excluded middle}\\
			True
		\end{align*}	
	}
	\item{
		$[(p \Rightarrow \lnot q) \land (r \Rightarrow q)] \Rightarrow (p \Rightarrow \lnot r)$
		\begin{align*}
			\lnot [(\lnot p \lor \lnot q) \land (\lnot r \land q)] & \lor \lnot p \lor \lnot r \tag{Material Implication}\\
			\lnot (\lnot p \lor \lnot q) \lor \lnot(\lnot r \land q) & \lor \lnot p \lor \lnot r \tag{De Morgan's Law}\\
			(p \land q) \lor (r \land \lnot q) & \lor \lnot p \lor \lnot r \tag{De Morgan's Law}\\
			\lnot p \lor (p \land q) \lor \lnot r \lor (r \land \lnot q) \tag{Commutative}\\
			[(\lnot p \lor p) \land (\lnot p \lor q)] \lor [(\lnot r \lor r) \land (\lnot r \lor \lnot q)] \tag{Distributive}\\
			[(True) \land (\lnot p \lor q)] \lor [(True) \land (\lnot r \lor \lnot q)] \tag{Excluded middle}\\
			\lnot p \lor q \lor \lnot r \lor \lnot q \tag{Excluded middle}\\
			True 
		\end{align*}
		}
	\item{
		$(p \Rightarrow q) \Rightarrow [\lnot(q \land r) \Rightarrow \lnot(p \land r)]$	
		\begin{align*}
			\lnot(\lnot p \lor q) \lor [\lnot \lnot(q \land r) \lor \lnot(p \land r)] \tag{De Morgan's Law}\\
			(p \land \lnot q) \lor (q \land r) \lor (\lnot p \lor \lnot r) \tag{De Morgan's Law + Negation}\\
			\lnot p \lor (p \land \lnot q) \lor \lnot r \lor (q \land r) \tag{Commutative}\\
			[(\lnot p \lor p) \land (\lnot p \lor \lnot q)] \lor [(\lnot r \lor q) \land (\lnot r \lor r)] \tag{Distributive}\\
			[(True) \land (\lnot p \lor \lnot q)] \lor [(\lnot r \lor q) \land (True)] \tag{Excluded middle}\\
			\lnot p \lor \lnot q \lor \lnot r \lor q \tag{Excluded middle}\\
			True
		\end{align*}
		}
	\item{
		$[(p \Rightarrow \lnot q) \land (\lnot r \lor q) \land r] \Rightarrow \lnot p$
		\begin{align*}
			\lnot [(\lnot p \lor \lnot q) \land (\lnot r \lor q) \land r] \lor \lnot p \tag{Material Implication}\\
			\lnot (\lnot p \lor \lnot q) \lor \lnot (\lnot r \lor q) \lor \lnot r \lor \lnot p \tag{De Morgan's Law}\\
			(p \land q) \lor (r \land \lnot q) \lor \lnot r \lor \lnot p \tag{De Morgan's Law}\\
			\lnot p \lor (p \land q) \lor \lnot r \lor (r \land \lnot q)  \tag{Commutative}\\
			[(\lnot p \lor p) \land (\lnot p \lor q)] \lor [(\lnot r \lor r) \land (\lnot r \lor \lnot q)] \tag{Distributive}\\
			[(True) \land (\lnot p \lor q)] \lor [(True) \land (\lnot r \lor \lnot q)] \tag{Excluded middle}\\
			\lnot p \lor q \lor \lnot r \lor \lnot q \tag{Excluded middle} \\
			True \tag{Excluded middle} \\
		\end{align*}
		}
	\end{enumerate}


\section{C}
\begin{enumerate}[label=\alph*.]
	\item{
			Proposition r means q is true if p(x) is true for one x. \\
			Proposition s means q is true if p(x) is true for all x.
		}
	\item{
			\begin{Dequation}
			\begin{align*}
				r & \Leftrightarrow (\forall x)(p(x) \Rightarrow q)\\
				\lnot r & \Leftrightarrow \lnot [(\forall x)(p(x) \Rightarrow q)]\\
						& \Leftrightarrow \exists x \lnot (p(x) \Rightarrow q) \tag{Quantifer Negation}\\
						& \Leftrightarrow \exists x \lnot (\lnot p(x) \lor q) \tag{Material Implication}\\
						& \Leftrightarrow \exists x (\lnot \lnot p(x) \land \lnot q) \tag{De Morgan's Law}\\
						& \Leftrightarrow \exists x (p(x) \land \lnot q) \tag{Double Negation}\\
			\end{align*}
			\end{Dequation}
		\begin{Dequation}
			\begin{align*}
				s & \Leftrightarrow ((\forall x)p(x)) \Rightarrow q \\
				\lnot s & \Leftrightarrow \lnot[((\forall x)p(x)) \Rightarrow q]\\
				& \Leftrightarrow \lnot[\lnot ((\forall x)p(x)) \lor q] \tag{Material Implication}\\
				& \Leftrightarrow ((\forall x)p(x)) \land \lnot q \tag{De Morgan's Law}
			\end{align*}
			\end{Dequation}
		}
	\item{
			s $\Rightarrow$ r is a tautology. If s is true, then r would be true because s requires p(x) to be true for all x while r only requires p(x) for one x to be true. If s is false, then the whole statement would be vacuously true.   
		}
\end{enumerate}

\section{D}
\begin{corollary*}
	The additive inverse is unique.
\end{corollary*}

\begin{proof}
Suppose $u,v$ are the additive inverse of x.
\[x + u = 0 \quad x + v = 0\]
\begin{Dequation}
\begin{align*}
	x + u & = x + v \tag{Transitive property}\\ 
	u + x & = v + x \tag{Commutative property}\\
	u & = v \tag{Theorem 1.1}
\end{align*}
\end{Dequation}
\end{proof}

\begin{corollary*}
	The vector $0$ is unique.
\end{corollary*}

\begin{proof}
Suppose $u,v \in V$ satisfies the "zero property", which is defined as:

\begin{Dequation}
\begin{align*}
	\forall x \in V \quad x + u & = x \Rightarrow v + u = v \\
	\forall x \in V \quad x + v & = x \Rightarrow u + v = u \\
	u = u + v & = v + u = v \tag{Transitive property}\\
	u & = v \tag{Theorem 1.1}
\end{align*}
\end{Dequation}
\end{proof}

\section{E}
\begin{theorem*}
	[1.2(c) In any vector space the following statements are true.]
			{$a\textbf{0 = 0}$ \\ $\forall a \in F \quad \textbf{0} \in \textbf{V}$ \\ \ul{Any scalar multiplied by the 0 vector will result in the 0 vector.}}
\end{theorem*}
\begin{proof}
\begin{Dequation}
\begin{align*}
	a \textbf{0} & = a(\textbf{0} + \textbf{0}) \tag{Identity element of addition} \\ 
	a \textbf{0} & = a\textbf{0} + a\textbf{0} \tag{Distributive} \\ 
	a \textbf{0} - a \textbf{0} & = a\textbf{0} + a\textbf{0} - a\textbf{0} \tag{Inverse element of addition} \\ 
	\textbf{0} & = a \textbf{0}
\end{align*}
\end{Dequation}
\end{proof}

\section{F}
Prove that diagonal matrices (as defined in your book in Example 3, Section 1.3) are symmetric.

\begin{proof}
	\-\
	\begin{center}
	Let $D$ equal a diagonal matrix \\
	$D$ is symmetric $\Leftrightarrow (\forall i,j)(D_{i,j} = D_{j,i})$\\
	By definition of diagonal matrix:\\
	When $i \neq j, D_{i,j} = D_{j,i} = 0$\\
	When $i = j, D_{i,j} = D_{j,i}$\\
	$\therefore (\forall i,j)(D_{i,j} = D_{j,i})$ so diagonal matrix is symmetric
\end{center}
\end{proof}

\section{G}
Prove that \\
$W_1 = \{(a_1, a_2,\cdots,a_n) \in F^n : a_1 + a_2 + \cdots + a_n= 0 \}$ \\
is a subspace of $F_n$, but\\
$W_2 =\{(a_1, a_2,\cdots,a_n) \in F^n : a_1 + a_2 + \cdots + a_n= 1\}$ \\
is not.
\begin{proof}
	Proof that $W_1$ is a subspace of $F^n$
	\begin{enumerate}[label=\alph*.]
		\item{
		$0 \in W_1$
		\begin{Dequation}
		\begin{align*}
			\text{Let} \ a_1, a_2, \cdots, a_n & = 0\\
			0 + 0 + \cdots + 0 & = 0 \\
			\text{So the 0 vector: }(0, 0, \cdots, 0) & \in W_1
		\end{align*}
		\end{Dequation}
		}
		\item{
		$X, Y \in W_1 \Rightarrow X + Y \in W_1$
		\begin{Dequation}
		\begin{align*}
			X = (x_1, x_2, \cdots, x_n) & \quad Y = (y_1, y_2, \cdots, y_n) \\
			X + Y & = (x_1 + y_1, x_2 + y_2, \cdots, x_n + y_n)\\
			\sum_{i = 1}^{n} x_i + y_i & = x_1 + y_1 + x_2 + y_2 + \cdots + x_n + y_n\\
			& = (x_1 + x_2 + \cdots + x_n) + (y_1 + y_2 + \cdots + y_n)\\
			& = 0 + 0 \\
			& = 0 \in W_1
		\end{align*}
		\end{Dequation}
		}
	\item{
	$c \in F, X \in W_1 \Rightarrow cX \in W_1$
	\begin{Dequation}
	\begin{align*}
		X & = (x_1, x_2, \cdots, x_n) \\
		cX & = (cx_1, cx_2, \cdots, cx_n) \\
		\sum_{i = 1}^{n} cx_i & = cx_1 +  cx_2 + \cdots + cx_n \\
		& = (x_1 + x_2 + \cdots + x_n) + (y_1 + y_2 + \cdots + y_n)\\
		& = c(x_1, x_2, \cdots, x_n) \\
		& = c(0) \\
		& = 0 \in W_1
	\end{align*}
	\end{Dequation}
	}
	\end{enumerate}
\end{proof}
\begin{proof}
$W_2 =\{(a_1, a_2,\cdots,a_n) \in F^n : a_1 + a_2 + \cdots + a_n= 1\}$ is not a subspace of $F^n$\\
If $W_2$ is a subspace, there must be closure under vector addition.
\begin{Dequation}
	\begin{align*}
			X = (x_1, x_2, \cdots, x_n) & \quad Y = (y_1, y_2, \cdots, y_n) \\
			X + Y & = (x_1 + y_1, x_2 + y_2, \cdots, x_n + y_n)\\
			\sum_{i = 1}^{n} x_i + y_i & = x_1 + y_1 + x_2 + y_2 + \cdots + x_n + y_n\\
			& = (x_1 + x_2 + \cdots + x_n) + (y_1 + y_2 + \cdots + y_n)\\
			& = 1 + 1 \\
			& = 2 \notin W_2 
	\end{align*}
\end{Dequation}
$\therefore W_2$ is not a subspace of $F^n$.\\
Additionally, there is no 0 vector in $W_2$. 0 vector for polynomial space only exiss if each component in a vector is 0. This is not possible in $W_2$ because the sum of the components must equal 1. 
\end{proof}

\section{H}
$W = \{f(x) \in P(F) : f(x) = 0 \text{ or } f(x) \text{ has degree } n\}$
\end{document}
