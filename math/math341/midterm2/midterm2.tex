\documentclass[11pt]{scrartcl}
\usepackage[sexy]{../../evan}
\usepackage{cmbright}
\usepackage{cancel}
\usepackage[T1]{fontenc}
%\usepackage{enumerate}
\usepackage[shortlabels]{enumitem}
\usepackage[utf8]{inputenc}
\usepackage[margin=1in]{geometry}
%\usepackage[pdfborder={0 0 0},colorlinks=true,citecolor=red{}]{hyperref}
\usepackage{amsmath}
\usepackage{amssymb}
\usepackage{setspace}
\usepackage{systeme}

\makeatletter
\newenvironment{Dequation}
{%
  \def\tagform@##1{%
    \maketag@@@{\makebox[0pt][r]{(\ignorespaces##1\unskip\@@italiccorr)}}}%
  \ignorespaces
  }
  {%
  \def\tagform@##1{\maketag@@@{(\ignorespaces##1\unskip\@@italiccorr)}}%
  \ignorespacesafterend
  }
\makeatother

\title{Math 341: Midterm 2}
\author{Daniel Ko}
\date{Spring 2020}

\begin{document}
\maketitle

%problem 1
\section{}
Let
\begin{equation}
	\mathbf{A} = \left [ \begin{array}{cc}
			a & b \\
			c & d
		\end{array} \right ],
	\mbox{ and  } \mathbf{b} = \left ( \begin{array}{c}
			e \\
			f
		\end{array} \right )
\end{equation}
\begin{enumerate}[label=\alph*.]
	\item{
	      Suppose that $a\neq 0 $, compute the solution of $\mathbf{Ax = b}$
	      using row reduction and provide the conditions on $a,b,c,d$ such that your computations are valid.
	      Express the result as a simplified expression. (\textbf{Hint:} recall that you can not divide by zero)
	      \begin{proof}
		      We perform reduced row echelon form (rref) on the augmented matrix
		      \begin{align*}
			      (A|b)=
			      \left[\begin{array}{cc|c}
					      a & b & e \\
					      c & d & f
				      \end{array}\right] \\
			      R_2 \leftarrow R_2 - \frac{c}{a} R_1
			      \left[\begin{array}{cc|c}
					      a & b                & e                \\
					      0 & d - \frac{cb}{a} & f - \frac{ce}{a}
				      \end{array}\right] \\
			      \left[\begin{array}{cc|c}
					      a & b               & e               \\
					      0 & \frac{ad-cb}{a} & \frac{af-ce}{a}
				      \end{array}\right] \\
			      R_2 \leftarrow \frac{a}{ad-cb}R_2 \quad \text{Assuming that $ab-cd \neq 0$} \quad
			      \left[\begin{array}{cc|c}
					      a & b & e                   \\
					      0 & 1 & \frac{af-ce}{ad-cb}
				      \end{array}\right] \\
			      R_1 \leftarrow R_1 - bR_2
			      \left[\begin{array}{cc|c}
					      a & 0 & e -b \frac{af-ce}{ad-cb} \\
					      0 & 1 & \frac{af-ce}{ad-cb}
				      \end{array}\right] \\
			      R_1 \leftarrow \frac{R_1}{a}
			      \left[\begin{array}{cc|c}
					      1 & 0 & \frac{1}{a}(e -b \frac{af-ce}{ad-cb}) \\
					      0 & 1 & \frac{af-ce}{ad-cb}
				      \end{array}\right] \\
			      \left[\begin{array}{cc|c}
					      1 & 0 & \frac{de-bf}{ad-cb} \\
					      0 & 1 & \frac{af-ce}{ad-cb}
				      \end{array}\right]
		      \end{align*}

		      \[x =
			      \begin{bmatrix}
				      \frac{de-bf}{ad-cb} \\
				      \frac{af-ce}{ad-cb}
			      \end{bmatrix} \quad \text{ where } ad-cb \neq 0
		      \]
	      \end{proof}
	      }
	\item{
	      If $a = 0$, and $c\neq 0$, is your above computation still valid?
	      How would you modify it? (explain briefly)
	      (\textbf{Hint:} recall that you can swap the equations and the result is the same)
	      \begin{proof}
		      If $a = 0$, and $c\neq 0$, then the above computation will not be valid as we
		      divided by $a$ multiple times when we computed the rref. I would swap the first
		      and second rows so that it would look like
		      \[
			      \left[\begin{array}{cc|c}
					      c & d & f \\
					      0 & b & e
				      \end{array}\right]
		      \]
		      and compute the rref, assuming that $b\neq0$. We obtain the rref,
		      \[
			      \left[\begin{array}{cc|c}
					      1 & 0 & \frac{bf-de}{bc} \\
					      0 & 1 & \frac{e}{b}
				      \end{array}\right]
		      \]
	      \end{proof}
	      }
	\item{
	      If $a = 0$, $c=0$, but $b \neq 0$, $d \neq 0$,
	      what are the conditions on $e$ and $f$ such that the system $\mathbf{Ax=b}$ has a solution?
	      Is the solution unique?  (\textbf{Hint:} recall that $\mathbf{Ax = b}$ has a solution if and only if
	      $\mathbf{b}$ can be written as a linear combination of the columns of $\mathbf{A}$)
	      \begin{proof}
		      If $a = 0$, $c=0$, $b \neq 0$, $d \neq 0$, we get the augmented matrix
		      \[
			      \left[\begin{array}{cc|c}
					      0 & b & e \\
					      0 & d & f
				      \end{array}\right]
		      \]
		      Performing row reduction,
		      \[
			      \left[\begin{array}{cc|c}
					      0 & 1 & \frac{e}{b} \\
					      0 & 1 & \frac{f}{d}
				      \end{array}\right]
		      \]
		      So the condition of the solution is, $$x_2 = \frac{e}{b} = \frac{f}{d}$$
		      Thus, there exists a infinite amount of solution.
	      \end{proof}
	      }
	\item{
	      Solve the system
	      \begin{equation}
		      \left [ \begin{array}{cc}
				      \sqrt{2}  & 3\sqrt{2} \\
				      2\sqrt{2} & \sqrt{2}
			      \end{array} \right ] \left ( \begin{array}{c}
				      x_1 \\
				      x_2
			      \end{array} \right ) = \left ( \begin{array}{c}
				      5\sqrt{2} \\
				      5\sqrt{2}
			      \end{array} \right ).
	      \end{equation}
	      (\textbf{Hint:} You may want to use the formula you just deduced)
	      }
	      \begin{proof}
		      \begin{align*}
			      x_1 & = \frac{de-bf}{ad-cb}                                                        \\
			          & = \frac{\sqrt2(5\sqrt2) - 3\sqrt2(5\sqrt2)}{\sqrt2\sqrt2 - 3\sqrt2(2\sqrt2)} \\
			          & = \frac{10 - 30}{2-12}                                                       \\
			          & = \frac{-20}{-10}                                                            \\
			          & = 2                                                                          \\
			          &                                                                              \\
			      x_2 & = \frac{\sqrt2(5\sqrt2) - 2\sqrt2(5\sqrt2)}{\sqrt2\sqrt2 - 3\sqrt2(2\sqrt2)} \\
			          & = \frac{10-20}{-10}                                                          \\
			          & = 1
		      \end{align*}
	      \end{proof}
\end{enumerate}

%problem 2
\section{}
Let
\begin{equation}
	\mathbf{A} = \left [ \begin{array}{cccc}
			0  & - \alpha & 2  & 0             \\
			1  & 1        & 0  & 1             \\
			2  & 2        & 2  & 3             \\
			-2 & -2       & 4  & 2\alpha       \\
			0  & \alpha   & -1 & 2\alpha + 1/2
		\end{array} \right ], \qquad  \mbox{and }
	\mathbf{b} =  \left [ \begin{array}{c}
			2          \\
			1          \\
			4          \\
			2 + \alpha \\
			2 \beta + \alpha -2
		\end{array} \right ]
\end{equation}
What are the conditions on $\alpha$ and $\beta$ such that the system $\mathbf{A} \mathbf{x} = \mathbf{b}$:
\begin{enumerate}[label=\alph*.]
	\item{
	      Has no solution?
	      \begin{proof}
		      We begin by putting the augmented matrix $(\mathbf{A}|\mathbf{b})$ in its reduced form.
		      \begin{align*}
			      (\mathbf{A}|\mathbf{b})=
			      \left [ \begin{array}{cccc|c}
					      0  & - \alpha & 2  & 0             & 2                  \\
					      1  & 1        & 0  & 1             & 1                  \\
					      2  & 2        & 2  & 3             & 4                  \\
					      -2 & -2       & 4  & 2\alpha       & 2 + \alpha         \\
					      0  & \alpha   & -1 & 2\alpha + 1/2 & 2\beta + \alpha -2
				      \end{array} \right ] \\
			      R_5 \leftarrow R_5 + R_1
			      \left [ \begin{array}{cccc|c}
					      0  & - \alpha & 2 & 0             & 2               \\
					      1  & 1        & 0 & 1             & 1               \\
					      2  & 2        & 2 & 3             & 4               \\
					      -2 & -2       & 4 & 2\alpha       & 2 + \alpha      \\
					      0  & 0        & 1 & 2\alpha + 1/2 & 2\beta + \alpha
				      \end{array} \right ] \\
			      R_3 \leftarrow R_3 - 2R_2
			      \left [ \begin{array}{cccc|c}
					      0  & - \alpha & 2 & 0             & 2               \\
					      1  & 1        & 0 & 1             & 1               \\
					      0  & 0        & 2 & 1             & 2               \\
					      -2 & -2       & 4 & 2\alpha       & 2 + \alpha      \\
					      0  & 0        & 1 & 2\alpha + 1/2 & 2\beta + \alpha
				      \end{array} \right ] \\
			      R_4 \leftarrow R_4 + 2 R_2
			      \left [ \begin{array}{cccc|c}
					      0 & - \alpha & 2 & 0             & 2               \\
					      1 & 1        & 0 & 1             & 1               \\
					      0 & 0        & 2 & 1             & 2               \\
					      0 & 0        & 4 & 2\alpha  +2   & 4 + \alpha      \\
					      0 & 0        & 1 & 2\alpha + 1/2 & 2\beta + \alpha
				      \end{array} \right ] \\
			      R_4 \leftarrow R_4 - 2R_3,R_5 \leftarrow R_5 - \frac12R_3
			      \left [ \begin{array}{cccc|c}
					      0 & - \alpha & 2 & 0       & 2                   \\
					      1 & 1        & 0 & 1       & 1                   \\
					      0 & 0        & 2 & 1       & 2                   \\
					      0 & 0        & 0 & 2\alpha & \alpha              \\
					      0 & 0        & 0 & 2\alpha & 2\beta + \alpha + 1
				      \end{array} \right ] \\
			      R_5 \leftarrow R_5 - R_4
			      \left [ \begin{array}{cccc|c}
					      0 & - \alpha & 2 & 0       & 2          \\
					      1 & 1        & 0 & 1       & 1          \\
					      0 & 0        & 2 & 1       & 2          \\
					      0 & 0        & 0 & 2\alpha & \alpha     \\
					      0 & 0        & 0 & 0       & 2\beta - 1
				      \end{array} \right ] \\
			      R_1 \leftrightarrow R_2
			      \left [ \begin{array}{cccc|c}
					      1 & 1        & 0 & 1       & 1          \\
					      0 & - \alpha & 2 & 0       & 2          \\
					      0 & 0        & 2 & 1       & 2          \\
					      0 & 0        & 0 & 2\alpha & \alpha     \\
					      0 & 0        & 0 & 0       & 2\beta - 1
				      \end{array} \right ]
		      \end{align*}
		      Thus this system will have no solution if $2\beta - 1 \neq 0$, which is when $\beta \neq \frac12$.
		      expain more...
	      \end{proof}
	      }
	\item{
	      Has an unique solution? Find the solution.
	      (\textbf{Hint:} you will need to row reduce the augmented system to echelon form,
	      and then use the theorems seen in class to impose the conditions on $\alpha$ and $\beta$).
	      \begin{proof}
		      Combining Theorem 3.10 and the corollary to Theorem 4.7, we get that a system has a
		      unique solution if and only if det$(\mathbf{A}) \neq 0$.
		      The determinant of an upper triangular matrix is the product of its diagonal
		      entries by property 4 of determinants in section 4.4.
		      %IF the above fact is not allowed, I will write out the full recursive definition 
		      So, this is when
		      \begin{align*}
			      1*-\alpha*2*2\alpha & \neq 0 \\
			      -4\alpha^2          & \neq 0 \\
			      \alpha              & \neq 0
		      \end{align*}
		      and from (a), $\beta = \frac12$. Combining these two conditions we get the following system,
		      \[
			      \left [ \begin{array}{cccc|c}
					      1 & 1        & 0 & 1       & 1      \\
					      0 & - \alpha & 2 & 0       & 2      \\
					      0 & 0        & 2 & 1       & 2      \\
					      0 & 0        & 0 & 2\alpha & \alpha \\
					      0 & 0        & 0 & 0       & 0
				      \end{array} \right ]
		      \]
		      So the generic solution will be
		      \[
			      x = \begin{bmatrix}
				      \frac12 + \frac{1}{2\alpha} \\
				      -\frac{1}{2\alpha}          \\
				      \frac34                     \\
				      \frac12                     \\
				      0
			      \end{bmatrix}
		      \]
	      \end{proof}
	      }

	\item{
	      Has infinite amount of solutions? Find the solution set in parametric form.
	      (\textbf{Hint:} You may have one equations for $\alpha$ and one for $\beta$ that have to be satisfied simultaneously).
		  \begin{proof}
			Having an infinite amount of solutions is by definition another way of saying
			that a system that is consistent and that the solutions are not unique.
			A system is consistent if $rank(\mathbf{A}) = rank(\mathbf{A}|\mathbf{b})$
			by Theorem 3.11. A solution is not unique if det$(\mathbf{A}) = 0$ by 
			Theorem 3.10 and the corollary to Theorem 4.7.

			  If det$(\mathbf{A}) = 0$, 
			  \begin{align*}
				1*-\alpha*2*2\alpha & = 0 \\
				-4\alpha^2          & = 0 \\
				\alpha              & = 0
				\end{align*}
			  Given that $\alpha = 0$, and that $\beta = \frac12$ from part (a)
			  we get the following system,

		      \[
			      \left [ \begin{array}{cccc|c}
					      1 & 1 & 0 & 1 & 1          \\
					      0 & 0 & 2 & 0 & 2          \\
					      0 & 0 & 2 & 1 & 2          \\
					      0 & 0 & 0 & 0 & 0          \\
					      0 & 0 & 0 & 0 & 0
				      \end{array} \right ]
			  \]
			  It is clear that  $rank(\mathbf{A}) = rank(\mathbf{A}|\mathbf{b})$ because 
			  $\mathbf{b}$ is a linear combination of the second and third column from 
			  $\mathbf{A}$. 
	      \end{proof}
	      }
\end{enumerate}


\section{}


\section{}

\end{document}