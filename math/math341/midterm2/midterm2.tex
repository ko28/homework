\documentclass[11pt]{scrartcl}
\usepackage[sexy]{../../evan}
\usepackage{cmbright}
\usepackage{cancel}
\usepackage[T1]{fontenc}
%\usepackage{enumerate}
\usepackage[shortlabels]{enumitem}
\usepackage[utf8]{inputenc}
\usepackage[margin=1in]{geometry}
%\usepackage[pdfborder={0 0 0},colorlinks=true,citecolor=red{}]{hyperref}
\usepackage{amsmath}
\usepackage{amssymb}
\usepackage{setspace}
\usepackage{systeme}

\makeatletter
\newenvironment{Dequation}
{%
  \def\tagform@##1{%
    \maketag@@@{\makebox[0pt][r]{(\ignorespaces##1\unskip\@@italiccorr)}}}%
  \ignorespaces
  }
  {%
  \def\tagform@##1{\maketag@@@{(\ignorespaces##1\unskip\@@italiccorr)}}%
  \ignorespacesafterend
  }
\makeatother

\title{Math 341: Midterm 2}
\author{Daniel Ko}
\date{Spring 2020}

\begin{document}
\maketitle

%problem 1
\section{}
Let
\begin{equation}
	\mathbf{A} = \left [ \begin{array}{cc}
			a & b \\
			c & d
		\end{array} \right ],
	\mbox{ and  } \mathbf{b} = \left ( \begin{array}{c}
			e \\
			f
		\end{array} \right )
\end{equation}
\begin{enumerate}[label=\alph*.]
	\item{
	      Suppose that $a\neq 0 $, compute the solution of $\mathbf{Ax = b}$
	      using row reduction and provide the conditions on $a,b,c,d$ such that your computations are valid.
	      Express the result as a simplified expression. (\textbf{Hint:} recall that you can not divide by zero)
	      \begin{proof}
		      We perform reduced row echelon form (rref) on the augmented matrix
		      \begin{align*}
			      (A|b)=
			      \left[\begin{array}{cc|c}
					      a & b & e \\
					      c & d & f
				      \end{array}\right] \\
			      R_2 \leftarrow R_2 - \frac{c}{a} R_1
			      \left[\begin{array}{cc|c}
					      a & b                & e                \\
					      0 & d - \frac{cb}{a} & f - \frac{ce}{a}
				      \end{array}\right] \\
			      \left[\begin{array}{cc|c}
					      a & b               & e               \\
					      0 & \frac{ad-cb}{a} & \frac{af-ce}{a}
				      \end{array}\right] \\
			      R_2 \leftarrow \frac{a}{ad-cb}R_2 \quad \text{Assuming that $ab-cd \neq 0$} \quad
			      \left[\begin{array}{cc|c}
					      a & b & e                   \\
					      0 & 1 & \frac{af-ce}{ad-cb}
					  \end{array}\right] \\
				R_1 \leftarrow R_1 - bR_2
				\left[\begin{array}{cc|c}
					a & 0 & e -b \frac{af-ce}{ad-cb} \\
					0 & 1 & \frac{af-ce}{ad-cb}
				\end{array}\right] \\
				R_1 \leftarrow \frac{R_1}{a}
				\left[\begin{array}{cc|c}
					1 & 0 & \frac{1}{a}(e -b \frac{af-ce}{ad-cb})\\
					0 & 1 & \frac{af-ce}{ad-cb}
				\end{array}\right] \\
				\left[\begin{array}{cc|c}
					1 & 0 & \frac{de-bf}{ad-cb} \\
					0 & 1 & \frac{af-ce}{ad-cb}
				\end{array}\right]
			  \end{align*}
			
			\[x = 
			\begin{bmatrix}
			\frac{de-bf}{ad-cb}\\
			\frac{af-ce}{ad-cb}
			\end{bmatrix} \quad \text{ where } ad-cb \neq 0
			\]
			
	      \end{proof}
	      }
	\item{
	      If $a = 0$, and $c\neq 0$, is your above computation still valid?
	      How would you modify it? (explain briefly)
	      (\textbf{Hint:} recall that you can swap the equations and the result is the same)
	      \begin{proof}
		      If $a = 0$, and $c\neq 0$, then the above computation will not be valid as we
		      divided by $a$ multiple times when we computed the rref. I would swap the first
		      and second rows so that it would look like
		      \[
			      \left[\begin{array}{cc|c}
					      c & d & f \\
					      0 & b & e
				      \end{array}\right]
		      \]
		      and compute the rref, using c as
	      \end{proof}
	      }
	\item{
	      If $a = 0$, $c=0$, but $b \neq 0$, $d \neq 0$,
	      what are the conditions on $e$ and $f$ such that the system $\mathbf{Ax=b}$ has a solution?
	      Is the solution unique?  (\textbf{Hint:} recall that $\mathbf{Ax = b}$ has a solution if and only if
	      $\mathbf{b}$ can be written as a linear combination of the columns of $\mathbf{A}$)
	      }
	\item{
	      Solve the system
	      \begin{equation}
		      \left [ \begin{array}{cc}
				      \sqrt{2}  & 3\sqrt{2} \\
				      2\sqrt{2} & \sqrt{2}
			      \end{array} \right ] \left ( \begin{array}{c}
				      x_1 \\
				      x_2
			      \end{array} \right ) = \left ( \begin{array}{c}
				      5\sqrt{2} \\
				      5\sqrt{2}
			      \end{array} \right ).
	      \end{equation}
	      (\textbf{Hint:} You may want to use the formula you just deduced)
	      }
	      \begin{proof}
		      \begin{align*}
			      x_1 & = \frac{de-bf}{ad-cb}                                                        \\
			          & = \frac{\sqrt2(5\sqrt2) - 3\sqrt2(5\sqrt2)}{\sqrt2\sqrt2 - 3\sqrt2(2\sqrt2)} \\
			          & = \frac{10 - 30}{2-12}                                                       \\
			          & = \frac{-20}{-10}                                                            \\
			          & = 2                                                                          \\
			          &                                                                              \\
			      x_2 & = \frac{\sqrt2(5\sqrt2) - 2\sqrt2(5\sqrt2)}{\sqrt2\sqrt2 - 3\sqrt2(2\sqrt2)} \\
			          & = \frac{10-20}{-10}                                                          \\
			          & = 1
		      \end{align*}
	      \end{proof}
\end{enumerate}




\section{}

\section{}


\section{}

\end{document}