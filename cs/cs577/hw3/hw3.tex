

\documentclass[11pt]{scrartcl}
\usepackage[sexy]{../evan}
\usepackage{cmbright}
\usepackage{cancel}
\usepackage[T1]{fontenc}
%\usepackage{enumerate}
\usepackage[shortlabels]{enumitem}
\usepackage[utf8]{inputenc}
\usepackage[margin=1in]{geometry}
%\usepackage[pdfborder={0 0 0},colorlinks=true,citecolor=red{}]{hyperref}
\usepackage{amsmath}
\usepackage{amssymb}
\usepackage{setspace}
\usepackage{systeme}
\usepackage[vlined]{algorithm2e}
\usepackage{mathtools}
\SetStartEndCondition{ }{}{}%
\SetKwProg{Fn}{def}{\string:}{}
\SetKwFunction{Range}{range}%%
\SetKw{KwTo}{in}\SetKwFor{For}{for}{\string:}{}%
\SetKwIF{If}{ElseIf}{Else}{if}{:}{elif}{else:}{}%
\SetKwFor{While}{while}{:}{fintq}%
\newcommand{\forcond}{$i=0$ \KwTo $n$}
\SetKwFunction{FRecurs}{FnRecursive}%
\AlgoDontDisplayBlockMarkers\SetAlgoNoEnd\SetAlgoNoLine%
\newcommand{\opt}{\text{OPT}}
\newcommand{\maxp}{\text{MaxPath}}
\newcommand{\ctn}{\text{MaxContains}}
\newcommand{\dctn}{\text{MaxDoesNotContain}}
\newcommand{\mpath}{\text{MinPath}}

\title{CS 577: HW 3}
\author{Daniel Ko}
\date{Summer 2020}

\begin{document}
\maketitle

%problem 1
\section{
  Kleinberg Chapter 6, Q14
 }

%problem 1a
\subsection{
	Suppose it is possible to choose a single path $P$ that is an $s-t$ path in
	each of the graphs $G_0, G_1, \cdots , G_b$. Give a polynomial-time algorithm
	to find the shortest such path.
}
%More formally, we assume that there exists a path $p_\phi$ such that $p_\phi \subseteq G_i$ for all $i$.
We define the set of edges that exist in all points in time as $$E^* = \bigcap_{i=0}^{b} E_i$$
Since our graph is unweighted, we can perform breadth first search to find the shortest path from
$s$ to $t$ on $G^* = (V, E^*)$. This will be $O(n)$, where $n$ is the number of vertices in graph $G$.
%We define $n$ to be size of the biggest $E_i$
%or more formally, $n = \max(\{|E_i| : 0 \leq i \leq b\})$.
%The complexity of computing $E^*$ is $\Theta(nb)$.
%The computing complexity of breadth first search is $\Theta(|V| + |E^*|)$.
%The total complexity would be $\Theta(nb + |V| + |E^*|) = O(n(b + |V|))$
%which is polynomial-time.

%We will assume that each of these graphs Gi is connected.
\iffalse
	We define $\delta(v,t)$ to be the length of the shortest
	path from $v$ to $t$.
	$$
		\delta(v,t) =
		\begin{cases}
			1                                                   & \textbf{when } (v, t) \in E^*             \\
			\infty                                              & \textbf{when } v \text{ has been visited} \\
			\min(\{ 1 + \delta(\phi,t) \mid (v,\phi) \in E^*\}) & \textbf{otherwise}                        \\
		\end{cases}
	$$
	Similiar to (6.23) from the book, we can modify the Bellman–Ford algorithm
	to find the shortest path from $s$ to $t$ that exists in all $G_i$. We define $\opt(i,v)$ to be the length of the shortest
	path from $v$ to $t$ using at most $i$ edges.
	$$\opt(i,v) =
		\begin{cases}
			\min \bigg(\opt(i-1,v),\min\Big(\{1 + \opt(i-1,w) \mid  (v,w) \in E^*\}\Big) \bigg) & \textbf{when } i > 0 \\
			0                                                                                   & \textbf{when } v = t \\
			\infty                                                                              & \textbf{otherwise}
		\end{cases}
	$$
\fi


%problem 1b
\subsection{
	Give a polynomial-time algorithm to find a sequence of paths
	$P_0, P_1, \cdots , P_b$ of minimum cost, where $P_i$ is an $s-t$ path in $G_i$ for
	$i = 0, 1, \cdots, b$.
}
For an $s-t$ path $P$ in one
of the graphs $G_i$, we define the length of $P$ to be simply the number of
edges in $P$, and we denote this to be $\ell(P)$.
Formally, we
define $\operatorname{changes}(P_{0}, P_{1}, \ldots, P_{b})$ to be the number of indices $i \ (0 \leq i \leq b - 1)$
for which $P_i \neq P_{i+1}$.
Fix a constant $K>0 .$ We define the cost of the sequence of paths
$P_{0}, P_{1}, \ldots, P_{b}$
\[
	\operatorname{cost}\left(P_{0}, P_{1}, \ldots, P_{b}\right)=
	\sum_{i=0}^{b} \ell\left(P_{i}\right)+K \cdot \operatorname{changes}\left(P_{0}, P_{1}, \ldots, P_{b}\right)
\]
\subsubsection{
	Set up the recursive formula and justify its correctness.
}

We define $\mpath(i)$ to be the recurrence for the a sequence of paths of minimum cost
in $G_0, \cdots, G_i$. Let the set of edges that exist from graphs $G_a, \cdots, G_b$
be defined as
$$E_{a,b} = \bigcap_{i=a}^{b} E_i$$
Let $\ell_E(E_{a,b})$ be the length of the shortest $s-t$ path in $E_{a,b}$. If a path does not exist
we define $\ell_E(E_{a,b}) = \infty$. \\
The main idea behind MinPath is that there are two major cases for the minimum sequence of paths.
The first major case is if there exists a single path for all graphs from $G_0$ to $G_i$. 
Then, we just calculate the length of this single path and multiply it by the amount of graphs 
we are considering. Note that since there are no changes we do not have to worry about the cost 
of changing paths, i.e. the $K$ value in the cost function. The second major case is 
if we need to change paths. We look at all places where the sequence of paths changes to a path
that is contained in $G_i$. This path change happens between the graph $\phi$ and $\phi + 1$.
We recursively calculate the minimum cost of paths between $0$ and $\phi$ and compute 
the cost of the path from $\phi + 1$ to $i$, which includes the cost of changing paths, $K$. 
We also add in a special condition when we consider this second case which is $i > 0$. This is because 
if $i=0$, there would be no switching paths as there is only one graph to consider. We need at least 
2 graphs for switching paths to make sense.
We formally define MinPath as 
$$
	\mpath(i) = \min \Bigg(\ell_E\Big(E_{0,i}\Big) \cdot \Big(i+1\Big),
	\min\Big(\Big\{\mpath(\phi)	+ \ell_E\Big(E_{\phi+1,i}\Big) \cdot \Big(i - \phi \Big)+ K
	 \mid (0 \leq \phi < i) \land (i > 0) \Big\}\Big) \Bigg)
$$
We assume that the min function for an empty set will return an empty set. Additionally,
if one of parameters of the min function is an empty set, it will return the other parameter. 
%We have a special case for $\mpath$ when $i=0$. It is trivial to see that when there is only one graph,
%the minimum cost will be the length of the shortest path of graph $G_0$.   
%$$
%	\mpath(0) = \ell_E\Big(E_{0,0}\Big)
%$$

\begin{proof}
	We show by strong induction that our recurrence relation is correct. Let $P(i)$ be the predicate,
	"MinPath correctly computes the sequence of paths of minimum cost in $G_0, \cdots, G_i$".
	We define $i \in \mathbb{N}$. \\
	\textbf{Base Case:} When $i=0$, it is trivial to see that the correct minimum cost is
	$\ell_E(E_{0,0})$ as there is only one graph to consider. We see that value in the left
	parameter of the min function evaluates to $\ell_E\big(E_{0,0}\big) \cdot \big(0+1\big) = \ell_E(E_{0,0})$.
	The right parameter never evaluates. Thus, the min of these two values is $\ell_E(E_{0,0})$,
	which is the correct minimum cost. Therefore, $P(0)$ holds.\\
	\textbf{Inductive step:} Suppose $P(i)$ holds for all $0 \leq i \leq k$. We show that $P(k + 1)$ holds.
	We have two major cases. The first case is if there exists a single minimum path from graphs
	$G_0, \cdots, G_{k+1}$.  Then the cost of this sequence will be $\sum_{i=0}^{k+1} \ell_E(E_{0,k+1}) =
		\ell_E(E_{0,k+1})((k+1)+1)$, which is exactly what is in the first parameter of our min function.
	The second case is if the minimum path at $G_{k+1}$ is not the same as all the paths before it.
	We check by exhaustion where the optimal change is, where $\phi$ is the index of the last graph
	where its minimum path is different than the minimum path at graph $k+1$. The cost of the entire path
	would be the cost of the sequence of minimum paths from graph $0$ to graph $\phi$,
	which would be $\mpath(\phi)$,
	plus the length of the minimum path in graph $k+1$ times the amount of graphs this path
	shows up in plus some constant $K$. If no such path exists from $\phi + 1$ to $k + 1$, then
	$\ell_E(E_{\phi+1,k+1}) = \infty$ as defined. Since $\phi \leq k$, by our strong
	inductive hypothesis $\mpath(\phi)$ is correct. We do this for all values of $\phi$ 
	and pick the smallest cost. We choose the smaller of both cases and
	return the minimum cost. \\
	By strong induction, we have proven that $P(i)$ holds for all $i \in \mathbb{N}$.
	Therefore, MinPath is correct.
\end{proof}

\subsubsection{
	Write the pseudocode for the iterative version of the algorithm to find the minimum
	cost. You are not required to write pseudocode to find the shortest path.
}
\begin{algorithm}
	\KwData{A list $G$, containing graphs $G_0, \cdots, G_b$}
	\KwResult{The minimum cost of the sequence of paths in $G_0, \cdots, G_b$}
	\Fn{minCost($G$)}{
		dp $\leftarrow$ a list contain the minimum cost for the sequence of paths from $G_0, \cdots, G_i$\\
		\For{$i$ from $0$ to $b$}{
			dp$[i] = $ \\$\min \Big( \ell_E(E_{0,i}) \cdot (i+1),
			\min (\{ \text{dp}[\phi]+ \ell_E (E_{\phi+1,i}) \cdot (i - \phi )+ K \mid 
			(0 \leq \phi < i) \land (i > 0) \})\Big)$
		}
		\Return{\text{\normalfont dp[b]}}
	}
\end{algorithm}



\subsubsection{
	Analyze the computing complexity.
}
We claim that the computing complexity is $O(b^3n^2)$.
\begin{proof}
	%Computing $\ell_E(E_{i,j})$ takes $n$ time as we can use breath first search.  
	Our for loop iterates $b$ times.
	Let the number of vertices in $G$ be $n$. The amount of edges 
	in each graph is at most $n^2$.
	Computing $E_{\alpha,\beta}$ take at most $bn^2$ time, i.e. comparing the intersection
	of $b$ graphs with $n^2$ edges. Computing the $\ell_E$ function takes $n$ time if we 
	use breadth first search. In total, computing $\ell_E(E_{\alpha,\beta})$ takes
	$bn^2 + n$. Inside the for loop, we compute $\ell_E(E_{\alpha,\beta})$ at most $b$ times.
	The rest of the computations take constant time. Hence our total time complexity is 
	$O(b(b(bn^2 + n))) = O(b^3n^2)$.
\end{proof}


%problem 2
\section{
  Given a	rooted tree $T = (V, E)$ and an integer $k$, find the largest possible
  number of disjoint paths in $T$, where each path has length $k$.
 }
\subsection{
	Set up the recursive formula and justify its correctness.
}
We define $\maxp(v)$ to be the recurrence for the maximum number of disjoint paths of size $k$ in a sub tree
of $T$ with root $v$. We consider two major cases, the maximum number of disjoint paths may or
may not contain $v$.

$$
	\maxp(v) = \max \Big(\dctn(v), \ctn(v, 0) \Big)
$$
In the case where maximum number of disjoint paths does not contain $v$, we define
$\dctn(v)$ to be the sum of
the maximum number of disjoint paths of size $k$ for each sub tree generated by the children of $v$,
i.e. the sub trees having $c$ as the root where $(v,c) \in E$.
We will define the set of child nodes of a vertex $v$ as $$C_v = \Big\{c \mid (v,c) \in E\Big\}$$
Adding up all the maximum paths for each sub tree
gives us the total amount of maximum paths for the entire tree.

$$
	\dctn(v) = \sum_{c \in C_v } \maxp(c)
$$
In the case where the maximum number of disjoint paths contains $r$, there are a couple of subcases.
We define $\ctn(v,\delta)$ to be the maximum number of disjoint paths of size $k$
in a sub tree of $T$ with root $v$, given that
$v$ is currently a part of a path of size $\delta$, where $\delta$ is bounded by $0 \leq \delta \leq k$.
The first subcase is if $v$ is a leaf node and the current size of the path it's on is less than $k$,
i.e $\delta < k$, there would be $0$ disjoint paths of size $k$ in this sub tree.
The second subcase is when $v$ is the last node in a path of size $k$, i.e. $\delta = k$. Then, the
maximum paths will be the sum of the maximum paths on the sub trees generated by the children of
$v$, or equivalently $\dctn(v)$, plus the path that $v$ is on.
The third subcase covers when $v$ is on some path that is not complete nor trivially ends on $v$.
The maximum number of disjoint paths in the sub tree of root $v$ currently a part of a path of size $\delta$,
will be the maximum number of disjoint paths of some sub tree $c \in C_v$ which would be
part of a path of size $\delta + 1$, plus the sum of the maximum number of disjoint paths for
the rest of the children of $v$. We try out all combinations of $c$ and pick the maximal one.

$$
	\ctn(v,\delta) =
	\begin{cases}
		0            & \textbf{when } \delta < k \textbf{ and } v \text{ is a leaf} \\
		\dctn(v) + 1 & \textbf{when } \delta = k                                    \\
		\max\bigg(\Big\{
		\ctn(c,\delta + 1)                                                          \\
		\qquad \qquad \qquad
		+ \displaystyle\sum_{c' \in C_v \setminus \{c\}} \maxp(c') \mid c \in C_v
		\Big\}\bigg) & \textbf{otherwise}                                           \\
	\end{cases}
$$

\begin{proof}
	We show by strong induction that our recurrence relation is correct. Let $P(n,k)$ be the predicate,
	"MaxPath correctly computes the maximum number of disjoint paths of size $k$ in a sub tree
	of $T$ which has $n$ number of nodes and where $v$ is the root node".
	We assume that the size of a path must be at least 1, as a path of 0 would lead to infinite amount of
	paths, and that a sub tree must contain at least one node, as our sub tree has a root.
	Hence, we define $n,k \in \mathbb{N}^+$. \\
	\textbf{Base Case:} When $n = 1, k = 1$, MaxPath returns 0. $\ctn(v,0)$ returns 0 because
	$0 < 1$ and $v$ is a leaf. $\dctn(v)$ returns 0 because $v$ has no children. The max of $\{0,0\}$
	equals $0$. It is clear to see that there are no paths for a tree that consists of a single node.
	Hence, $P(1,1)$ holds.\\
	\textbf{Inductive step for number of nodes:} Suppose $P(\eta,\kappa)$ holds for all $n \leq \eta$ and $k \leq \kappa$.
	We show that $P(\eta + 1,\kappa)$ holds. MaxPath(v) calls $\ctn(v,0)$ and $\dctn(v)$. Since $v$
	is not a leaf and $0 \neq k$, third condition holds in $\ctn$.
	Notice that the number of nodes the sub tree where the root is $c$ will have strictly less nodes than $n + 1$,
	which is amount of nodes in the sub tree where the root is $v$ because $c$ is a child of $v$.
	%Given the previous statement and $\kappa-1 \leq k$, 
	Hence, $\ctn(c, 1)$ holds by our inductive hypothesis.
	Without loss of generality, $\maxp(c')$ and $\dctn(v)$ holds by our inductive hypothesis.
	Therefore, $P(\eta + 1,\kappa)$ holds.\\
	%\ctn is true by ih , tree of c is less
	%\dctn is true by ih 
	\textbf{Inductive step for length of path:} Suppose $P(\eta,\kappa)$ holds for all $n \leq \eta$ and
	$k \leq \kappa$. We show that $P(\eta,\kappa + 1)$ holds. MaxPath(v) calls $\ctn(v,0)$ and $\dctn(v)$.
	We continue with the third condition in $\ctn(v,0)$.
	Notice that $\ctn(c, 1)$ is equivalent to $\ctn(c,0)$ if we fix $k = \kappa$.
	Since $\kappa \leq \kappa + 1$, $\ctn(c,0)$ where $k = \kappa$ and subsequently
	$\ctn(c, 1)$ where $k = \kappa + 1$ holds by our inductive hypothesis.
	Without loss of generality, $\maxp(c')$ and $\dctn(v)$ holds by our inductive hypothesis
	because they will eventually reach $\ctn(v', 1)$. Therefore, $P(\eta,\kappa+ 1)$ holds.\\
	By strong induction, we have proven that $P(n,k)$ holds for all $n,k \in \mathbb{N}^+$.
	Therefore, MaxPath is correct.
\end{proof}
\pagebreak 



\subsection{
	Write the pseudocode for the iterative version of the algorithm to find the maximum
	number of players that can play at the same time.
}
The main idea of this algorithm is to compute bottom up MaxPath. We iterate over all
vertices in the tree using post order traversal. We compute all MaxContains values
for that vertex and possible size of the paths that they are on. All the 
recursive values needed in the recurrence relationship will be computed by the time 
its needed.  

\begin{algorithm}
	\KwData{$T = (V, E)$, a tree with root $r$; $k$, length of path}
	\KwResult{Maximum number of disjoint paths of size $k$ in the tree $T$}
	\Fn{max($T, k$)}{
		contains $\leftarrow$ a map representing memoized values of MaxContains, where $<v,\delta>$ is the key\\
		notContains $\leftarrow$ a map representing memoized values MaxDoesNotContain, where node $v$ is the key\\
		maxPaths $\leftarrow$ a map representing memoized values MaxPath, where node $v$ is the key\\
		\For{$v \in V$ in post order traversal}{
			\#No paths unless this is the last edge needed to create a path \\
			\If{$v$ is a leaf}{
				maxPaths.insert(v, 0)\\
				notContains.insert(v, 0)\\
				\For{i from 0 to k - 1}{
					contains.insert(<v,i>, 0)
				}
				contains.insert(<v,k>, 1)\\
			}
			\# Computing MaxDoesNotContain(v) and MaxContains(v,$\delta$)  \\
			\Else{
				notContainVal = 0\\
				\For{each child $c$ of $v$}{
					notContainVal+=maxPaths.get(c)
				}
				notContains.insert(v,notContainVal)\\
				\For{i from 0 to k-1}{
					containsVal $\leftarrow$ a empty list\\
					\For{each child $c$ of $v$}{
						\#Where $c' \in C_v \setminus\{c\}$\\
						val = contains.get(c,i+1) + $\sum$ maxPaths.get(c')\\
						containsVal.insert(val)
					}
					contains.insert(<v,i>, max(containsVal))
				}
				contains.insert(<v,k>, 1+ notContains.get(v))\\
				maxPaths.insert(v, max(contains.get(<v,0>),notContains.get(v)))
			}
		}
		\Return{maxPaths.get(r)}
	}
\end{algorithm}



\subsection{
	Analyze the computing complexity.
}
We claim the time complexity for max is $O(nk)$.
\begin{proof}
	Let $n$ be the number of vertices in $T$. The outer for loop runs $n$ times. 
	Inside the if statement, there is a for loop that runs $k$ times. All other 
	computations in the if statement runs in constant time. This leads to computing complexity 
	of $O(nk)$ for the for loop and the if statement. \\\-\\
	In the else statement, we access 
	the child nodes of a vertex. Notice that no child vertex will be accessed twice,
	leading to a complexity of $O(n)$ for the for loop that goes through all the vertices 
	and the for loop accessing the children. The total complexity of outer for loop and
	the else statement 
	is $O(2n + nk) = O(nk)$, where $2n$ comes from the 2 for loops for accessing the children 
	and $nk$ comes from the outer for loop, n, and the inner for loop, k. \\
	In either the if statement or the else statement, the time complexity is $O(nk)$.
	Therefore, the time complexity for max is $O(nk)$.
\end{proof}

\end{document}

