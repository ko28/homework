\documentclass[11pt]{scrartcl}
\usepackage[sexy]{../../evan}
\usepackage{cmbright}
\usepackage{cancel}
\usepackage[T1]{fontenc}
%\usepackage{enumerate}
\usepackage[shortlabels]{enumitem}
\usepackage[utf8]{inputenc}
\usepackage[margin=1in]{geometry}
%\usepackage[pdfborder={0 0 0},colorlinks=true,citecolor=red{}]{hyperref}
\usepackage{amsmath}
\usepackage{amssymb}
\usepackage{setspace}
\usepackage{systeme}

\makeatletter
\newenvironment{Dequation}
{%
	\def\tagform@##1{%
	\maketag@@@{\makebox[0pt][r]{(\ignorespaces##1\unskip\@@italiccorr)}}}%
	\ignorespaces
}
{%
	\def\tagform@##1{\maketag@@@{(\ignorespaces##1\unskip\@@italiccorr)}}%
	\ignorespacesafterend
}
\makeatother

\title{Math 240: Homework 5}
\author{Daniel Ko}
\date{Spring 2020}

\begin{document}
\maketitle

\section{1}
Fix positive integers $a$ and $b$. Here's an inductive definition of a set $S$:\\
Foundation rule: $a, b \in S$\\
Constructor rule: If $m, n \in S$, then $m - n \in S$.
\begin{enumerate}[label=\alph*.]
	\item{
			Suppose $h$ is a common factor of $a$ and $b$. Use the exclusion rule to prove that for every $n \in S$, $h$ divides $n$.
			\begin{proof}
				\-\\
				There must be integers $c_a$ and $c_b$ such that $a = hc_a$ and $b = hc_b$. By the contructor rule, 
				$a - b \in S$ where $a - b = hc_a - hc_b = h(c_a - c_b)$. Thus, all other generated elements will have $h$ as a common factor because
				of the exclusion rule. Therefore, by the exclusion rule we have proven that for every $n \in S$, $h$ divides $n$.
			\end{proof}
	\item{
			Suppose $k \in S$ is a positive integer which is not a factor of $a$. Prove that there is some $l \in S$ such that $0 < l < k$.
			\begin{proof}
			\-\\
			Consider the sequence: $\{a, a - k, a - 2k, \cdots \}$. Let this sequence be a subset of $\mathbb{N}$. 
			It is clear that all elements in this sequence are elements of $S$ by the constructor rule.
			By the well ordering principle, there exists a smallest element in this sequence. An $l$ that always fits the criteria of $0 < l < k$ is the 
			smallest element in this sequence. $l$ must be less than $k$ because if $l$ wasn't less than $k$ then $l$ would not be the smallest element in the sequence.
			Let $a - nk$ be the first negative term that is generated. Then, our sequence must stop at $a - (n + 1)k$ and this would be the smallest
			element in this set by the well ordering principle. Since, $a$ is not a factor of $k$, $l$ cannot be $0$. Therefore, there exists an $l$
			such that $0 < l < k$.
			\end{proof}
		}
	\item{
			Use (b) and the above fact to prove that there is some positive	integer in S which is a common factor of a and b.

			 \begin{proof}
			 \-\\
			 Consider this strictly decreasing sequence:
			 $\{(a + b), (a + b) - k, (a + b) - 2k, \cdots\}$ which is a subset of $\mathbb{N}$. By the well ordering principle, this sequence
			 must be finite, such that it ends at $(a + b) -nk > 0$ and $(a + b) - (n+1)k < 0$, where $n \in \mathbb{N}$.
			 Suppose there is no positive integer, $l$, in S that is a common factor of $a$ and $b$. 
			 By construction of our sequence $a + b - (n+1)k < l$. By (b) $l$ such that,
			 $ 0 < l < k_{!a}$ and $0 < l < k_{!b}$, where $k_{!a}$ and $k_{!b}$ are integers that are not factors of a and b respectively. 
			 However, this leads to a contradiction because if this were true, then $l$ would be a common factor of $a$ and $b$.
			 Thus, there exists a positive interger in S which is a common factor of a and b.
			 \end{proof}

		}
	\item{
			Use (a) and (c) to conclude that S contains gcd(a,b)
			\begin{proof}
				From part c, we know there is a positive integer in S that is a common divisor of a and b.
				From part a, we know that this will divide all $n \in S$. Assume that there is a gcd(a,b) that is greater than the common divisor 
				in $S$. This must mean gcd(a,b) $\neq m - n$, where $m,n \in S$. However, this is a contradiction because all common divisors must
				be generated using the constructor rule because it is a multiple of all $n \in S$. Thus, S must contain gcd(a,b).
			\end{proof}
		}
\end{enumerate}
\section{2}
\begin{enumerate}[label=\alph*.]
	\item{
			Prove by structural induction that for all $x \in \{0, 1\}^*, \lambda x = x$.
			\begin{proof}
			\-\\
			We proceed by induction on $x$. Let $P(x)$ be the predicate "$x \in \{0,1\}^*, \lambda x = x$".\\
			\-\\
			Base case: Show $P(\lambda)$ holds.\\
			$P(\lambda) = \lambda \lambda = \lambda$ by foundation rule in concatenation.\\
			\-\\
			Inductive step: If $P(x)$ holds, then $P(xi)$ holds, where $i \in \{0,1\}$\\
			Fix $x \in \{0, 1\}^*$ and $i \in \{0, 1\}$ and assume that $P(x)$ holds.
			We show that $P(xi)$ holds.
			\begin{Dequation}
				\begin{align*}
					P(xi) &= \lambda xi \\
						  &= xi \text { by inductive hypothesis} 
				\end{align*}
			\end{Dequation}
			\end{proof}
		}
	\item{
			Consider the function reverse: $\{0, 1\}^* \rightarrow \{0, 1\}^*$ which reverses a binary string\\
			e.g. reverse(01001) = 10010. Give an inductive definition of reverse.
			\begin{proof}
				\-\\
				Foundation rule: reverse(i) = i, where $i \in \{\lambda, 0, 1\}$. \\
				Constructor rule: reverse(xi) = ireverse(x), where $x \in \{0, 1\}^*$ 
			\end{proof}
		}
	\item{
		Using your inductive definition, prove that for all $x, y \in \{0, 1\}^*$, reverse(xy) = reverse(y)reverse(x).
		\begin{proof}
			\-\\
			Let $y \in \{0, 1\}^*$. $y_n$ repesents the nth digit and $y_0^{n-4}$ represents the substring from 0 to $n - 4$.
			Let the length of $y$ be n.
		\begin{Dequation}
		\begin{align*}
			\text{reverse}(xy) &= y_n\text{reverse}(xy_0^{n-1})\\
							   &= y_{n_1}y_{n-1}\text{reverse}(xy_0^{n-2})\\
							   & \cdots \\ 
							   &= y_ny_{n-1}y_{n-2} \cdots y_{1} y_{0} \text{reverse}(x)\\
							   &= \text{reverse}(y_n)y_{n-1}y_{n-2} \cdots y_{1} y_{0} \text{reverse}(x)\\
							   &= \text{reverse}(y_{n-1}y_n)y_{n-2} \cdots y_{1} y_{0} \text{reverse}(x)\\
							   & \cdots \\ 
							   &= \text{reverse}(y) \text{reverse}(x)
		\end{align*}
		\end{Dequation}
		\end{proof}
		}
\end{enumerate}
\end{document}
