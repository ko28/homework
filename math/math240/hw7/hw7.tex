\documentclass[11pt]{scrartcl}
\usepackage[sexy]{../../evan}
\usepackage{cmbright}
\usepackage{cancel}
\usepackage[T1]{fontenc}
%\usepackage{enumerate}
\usepackage[shortlabels]{enumitem}
\usepackage[utf8]{inputenc}
\usepackage[margin=1in]{geometry}
%\usepackage[pdfborder={0 0 0},colorlinks=true,citecolor=red{}]{hyperref}
\usepackage{amsmath}
\usepackage{amssymb}
\usepackage{setspace}
\usepackage{systeme}

\makeatletter
\newenvironment{Dequation}
{%
	\def\tagform@##1{%
	\maketag@@@{\makebox[0pt][r]{(\ignorespaces##1\unskip\@@italiccorr)}}}%
	\ignorespaces
}
{%
	\def\tagform@##1{\maketag@@@{(\ignorespaces##1\unskip\@@italiccorr)}}%
	\ignorespacesafterend
}
\makeatother

\title{Math 240: Homework 7}
\author{Daniel Ko}
\date{Spring 2020}

\begin{document}
\maketitle

\section{1}
Solve the following recurrence relations. Show your work. If you wish,
you may use the formulas we stated in lecture, without proof
\begin{enumerate}[label=\alph*.]
	\item{
		$a_0 = 1$ and $a_n = 3a_{n-1} + 2$ for $n \geq 1$
		\begin{proof}
			Given any recurrence relations in this form $a_n = ra_{n-1} + c$, we can rewrite the relation as
			$a_{n}=r^{n} a_{0}+c \frac{r^{n}-1}{r-1}$ if $r \neq 1$.
			\begin{align*}
				a_{n} & = 3^{n}(1) + 2 \frac{3^{n}-1}{3-1} \\
				& = 3^{n} + 3^{n}-1\\
				& = 2(3^{n}) -1
			\end{align*}
		\end{proof}
	}
	\item{
		$a_{0}=3, a_{1}=6$ and $a_{n}=a_{n-1}+6 a_{n-2}$ for $n \geq 2$
		\begin{proof}
			We can write the above recurrence relation as $r^2 - r - 6 = 0$. Solving for $r$ gets us $r = 3, r = -2$.
			We use the equation given in the lecture for linear homogeneous recurrence relations of degree two.
			\begin{align*}
			a_n & = \alpha(3)^n + \beta(-2)^n\\
			a_{0}=3 \Rightarrow 3 & = \alpha + \beta \\
			a_{1}=6 \Rightarrow 6 & = \alpha (3) + \beta (-2)\\
			\alpha = \frac{12}{5} \quad & \quad \beta = \frac{3}{5}\\ 
			a_n & = \frac{12}{5} 3^n + \frac{3}{5} (-2)^n 
			\end{align*}
		\end{proof}
	}
	\item{
		$a_{0}=5, a_{1}=-1$ and $a_{n}=a_{n-2}$ for $n \geq 2$
		\begin{proof}
			We can write the above recurrence relation as $r^2 - 1 = 0$. Solving for $r$ gets us $r = \pm 1$.
			We use the equation given in the lecture for linear homogeneous recurrence relations of degree two.
			\begin{align*}
			a_n & = \alpha(-1)^n + \beta(1)^n\\
			a_{0}=5 \Rightarrow 5 & = \alpha + \beta \\
			a_{1}=-1 \Rightarrow -1 & = -\alpha + \beta\\
			\alpha = 3 \quad & \quad \beta = 2\\ 
			a_n & = 3(-1)^n + 2(1)^n
			\end{align*}
		\end{proof}
	}
\end{enumerate}

\section{2}
\begin{enumerate}[label=\alph*.]
	\item{
		Prove that every string $\sigma$ of length $n$ with an even number of 0s has exactly one of the following forms:
		\begin{enumerate}[i]
			\item{
				$\sigma=\tau 1$ where $\tau$ is a string of length $n-1$ with an even number
				of $0$s.\\\-\\
				When we concatenate a 1 to $\tau$, the total number of 0s in $\sigma$ is the same as $\tau$.
				So $\sigma$ will have an even number of 0s.
			}
			\item{
				$\sigma=\tau 0$ where $\tau$ is a string of length $n-1$ with an odd number
				of 0s.\\\-\\
				When we concatenate a 0 to $\tau$, the total number of 0s in $\sigma$ is one more than $\tau$.
				An even number plus one is an odd number by definition.
				$\tau$ has an even number of 0's, so $\sigma$ will have an odd number of 0s.
			}
		\end{enumerate}
	}
	\item{
		Use (a) to find the number of binary strings of length n with an
		even number of 0s, as a function of n.\\\-\\
		We will assume that 0 is an even number, so for a number with no 0s will have an even number of 0s, including the empty string.
		From part (a) we can create a recurrence relation: $a_n = 2(a_{n-1})$. This because there are two cases when we concatenate a 
		binary string to $a_{n-1}$. If $a_{n-1}$ has an even number of 0s then we then we append a 1 to the end. 
		If $a_{n-1}$ has an odd number of 0s then we then we append a 0 to the end. We know that $a_0 = 1$ and $a_1 = 1$ because the 
		empty string and string with no zeros will have an even amount of zeros by our assumption. 
		We can write the above recurrence relation as $r^2 - 2r = 0$. Solving for $r$ gets us $r = 0, r = 2$.
		We use the equation given in the lecture for linear homogeneous recurrence relations of degree two.
		\begin{align*}
		a_n & = \alpha(0)^n + \beta(2)^n\\
		a_{0}=1 \Rightarrow 1 & = \alpha + \beta \\
		a_{1}=1 \Rightarrow 1 & = -\alpha0 + \beta2 = \beta2\\
		\alpha =\frac12 \quad & \quad \beta = \frac12\\ 
		a_n & = \frac12(0)^n + \frac12(2)^n
		\end{align*}
	}
\end{enumerate}


\section{3}
\begin{align*}
	H'_{n+1} & = H'_{n} + (n+1)^2 + H'_{n}\\
	& = 2H'_{n} + (n+1)^2\\
\end{align*}

\section{4}
\begin{enumerate}[label=\alph*.]
	\item{
	Prove that $5x + 4$ is $O(x^2)$
	\begin{proof}
		We need to choose $C, k > 0$ such that for every $x > k$
		\[
			5x + 4 \leq Cx^2
		\]
		Take $C = 9$ and $k = 2$. Then,
		\begin{align*}
			9x^2 & = 5x^2 + 4x^2\\
			& > 5x + 4 \qquad \qquad x > 1
		\end{align*}
		as desired. 
	\end{proof}
	}
	\item{
		Prove that $x^2$ is not $\Theta(5x + 4)$
		\begin{proof}
		To prove the above statement, we can prove that $x^2$ is not $O(5x+4)$.
		We need to show that for every $C, k > 0$, there is some $x > k$ such that 
		$x^2 > O(5x+4)$.\\
		Suppose we are given $C, k > 0$. Consider $x = \text{max}\{9C, k\} + 1$.\\
		Then $x>k$ and 
		\begin{align*}
			x^2 & > 9Cx & x > 9C \text{ and } x > 0\\
			& = 5Cx + 4Cx\\
			& > 5Cx + 4C & x > 1 \text{ and } C > 0\\
			& = C(5x+4)
		\end{align*}
		as desired. Since $x^2$ is not $O(5x+4)$, by definition $x^2$ is not $\Theta(5x + 4)$.
		\end{proof}
	}
\end{enumerate}
\end{document}
