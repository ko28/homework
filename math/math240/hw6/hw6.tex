\documentclass[11pt]{scrartcl}
\usepackage[sexy]{../../evan}
\usepackage{cmbright}
\usepackage{cancel}
\usepackage[T1]{fontenc}
%\usepackage{enumerate}
\usepackage[shortlabels]{enumitem}
\usepackage[utf8]{inputenc}
\usepackage[margin=1in]{geometry}
%\usepackage[pdfborder={0 0 0},colorlinks=true,citecolor=red{}]{hyperref}
\usepackage{amsmath}
\usepackage{amssymb}
\usepackage{setspace}
\usepackage{systeme}

\makeatletter
\newenvironment{Dequation}
{%
	\def\tagform@##1{%
	\maketag@@@{\makebox[0pt][r]{(\ignorespaces##1\unskip\@@italiccorr)}}}%
	\ignorespaces
}
{%
	\def\tagform@##1{\maketag@@@{(\ignorespaces##1\unskip\@@italiccorr)}}%
	\ignorespacesafterend
}
\makeatother

\title{Math 240: Homework 6}
\author{Daniel Ko}
\date{Spring 2020}

\begin{document}
\maketitle

\section{1}
\begin{enumerate}[label=\alph*.]
	\item{
		Prove that the above algorithm satisfies partial correctness.
		\begin{proof}
			Let $p$ be the proposition
			$a = dq + r \land r \geq 0$.\\
			Let's prove that $p$ is the loop invarient by induction.
			\begin{enumerate}[label=\roman*.]
			\item{
				Base case: When the loop condition is tested for the first time, $p$ holds\\
				\-\\
				$r=a$ and $q=0$ before the loop is run. 
				When the loop condition is first tested $a = dq + r$ holds because $q=0$, so
				$a = 0 + r = r$. \\
				$r \geq 0$ also holds because $a \in \mathbb{N}$. 
			}
			\item{
				Inductive step: If $p$ and the loop condition both hold at the beginning of an
				iteration of the loop, then $p$ holds right after said iteration\\
				\-\\
				Let $a = dq + r \land r \geq 0$ at some $n-1$ iteration of the loop (inductive hypothesis).
				Let $r_n$ and $q_n$ be the value of $r$ and $q$ at the $n$th iteration respectively. \\
				So at the $n$th iteration, $a=dq_n + r_n$. We know $q_n = q + 1$ and $r_n = r - d$.
				\begin{align*}
				a & = dq_n + r_n\\	
				& = d(q + 1) + r - d\\
				& = dq + d + r - d \\
				& = dq + r 
				\end{align*}
				We know that $r_n = r - d$. 
				Our loop condition is $r \geq d$, thus $r_n \geq 0$.
			}
			\end{enumerate}
		Therefore, $p$ is a loop invarient. \\
		Now we must show that loop invariant holds
		and the loop condition fails, then the output is correct. Suppose the loop halts at the 
		$n$th iteration, which is when $r < d$. 
		By our loop invarient, we have that $r \geq 0$. Combining the two statements
		above, we get $0 \leq r < d$. Also by our loop invariant, we have $a = dq + r$. 
		Therefore, this algorithm satisfies partial correctness.
		\end{proof}
	}
	\item{
		Prove that the above algorithm satisfies termination.
		\begin{proof}
			Our termination condition is $r < d$. $r$ is set to $a$ then decreases by 
			$d$ after each iteration. Since $d \in \mathbb{N}$, the sequence of the values of $r$ is 
			strictly decreasing. We also know that $r \geq 0$ by our loop invarient.
			Therefore, by the well-ordering principle, the algorithm satisfies termination.
		\end{proof}
	}
\end{enumerate}


\section{2}
Prove that for all $n \in \mathbb{N}$, SQ($n$) halts and returns $n^2$.
\begin{proof}
	We proceed by strong induction on $n$. Let P($n$) be the predicate 
	“SQ($n$) halts and returns $n^2$”. Domain: $\mathbb{N}$.
	\begin{enumerate}[label=\alph*.]
		\item{
			Base case:\\
			We prove P($0$). Observe that P($0$) halts and returns $0$, which is equal to $0^2$.
		}
		\item{
			Inductive step:\\
			Suppose $n \in \mathbb{N}$ and P holds for all integers between 0 and $n$, inclusive. 
			We show that P($n+1$) holds.
			\begin{align*}
				\text{SQ}(n+1)  & = \text{SQ}(n-1) + 2n - 1 \\
								& = (n-1)^2 + 2n - 1 \tag{Strong inductive hypothesis}\\
								& = n^2 -2n + 1 + 2n -1\\
								& = n^2
			\end{align*}
			By the strong inductive hypothesis P($n-1$), SQ($n-1$) halts. Thus, SQ($n+1$) halts.
		}
	\end{enumerate}
	Therefore, by strong induction we have proved that for all $n \in \mathbb{N}$, SQ($n$) halts and returns $n^2$.
\end{proof}

\section{3}
Prove that POW is correct.
\begin{proof}
	We proceed by strong induction on $b \in \mathbb{N}$. Let $P(b)$ be the following predicate 
	"$POW(a,b)$ halts and returns $a^b$, where $a$ is a fixed nonzero integer".
	\begin{enumerate}[label=\alph*.]
		\item{
			Base case:\\
			We prove $P(0)$. $POW(a,0) = 1$.
			Observe that P($0$) halts and returns $1$, which is equal to $a^0$.
		}
		\item{
			Inductive step:\\
			Suppose $n \in \mathbb{N}$ and P holds for all integers between 0 and $n$, inclusive. 
			We show that P($n+1$) holds.\\
			Case 1: $n+1$ is odd
			\begin{align*}
			POW(a,n+1) & = a \cdot (POW(a, \lfloor (n+1)/2 \rfloor))^2\\
			& = a \cdot (a^{\lfloor (n+1)/2 \rfloor)})^2 \tag{Strong inductive hypothesis}\\
			& = a \cdot (a^{n/2})^2 \tag{n is even}\\
			& = a \cdot a^n \\
			& = a^{n+1}
			\end{align*}
			We know that $(POW(a, \lfloor (n+1)/2 \rfloor))$ halts by our strong induction hypothesis so  
			$POW(a,n+1)$ halts as well.\\
			\-\\
			Case 2: if $n+1 > 0$ and $n+1$ is even
			\begin{align*}
				POW(a,n+1) & = (POW(a, \lfloor (n+1)/2 \rfloor))^2\\
				& = (a^{\lfloor (n+1)/2 \rfloor})^2 \tag{Strong inductive hypothesis}\\
				& = (a^{(n+1)/2})^2  \tag{n + 1 is even}\\
				& = a^{n+1}
			\end{align*}
			We know that $(POW(a, \lfloor (n+1)/2 \rfloor))$ halts by our strong induction hypothesis so  
			$POW(a,n+1)$ halts as well.\\
		}
		Therefore, by strong induction we have proved that $POW$ is correct.
	\end{enumerate}
\end{proof}

\end{document}
