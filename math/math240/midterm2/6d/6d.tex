\documentclass[11pt]{scrartcl}
\usepackage[sexy]{../../../evan}
\usepackage{cmbright}
\usepackage{cancel}
\usepackage[T1]{fontenc}
%\usepackage{enumerate}
\usepackage[shortlabels]{enumitem}
\usepackage[utf8]{inputenc}
\usepackage[margin=1in]{geometry}
%\usepackage[pdfborder={0 0 0},colorlinks=true,citecolor=red{}]{hyperref}
\usepackage{amsmath}
\usepackage{amssymb}
\usepackage{setspace}
\usepackage{systeme}
\usepackage{algorithmic}

\makeatletter
\newenvironment{Dequation}
{%
	\def\tagform@##1{%
	\maketag@@@{\makebox[0pt][r]{(\ignorespaces##1\unskip\@@italiccorr)}}}%
	\ignorespaces
}
{%
	\def\tagform@##1{\maketag@@@{(\ignorespaces##1\unskip\@@italiccorr)}}%
	\ignorespacesafterend
}
\makeatother

\title{Math 240: Midterm 2 Q6d}
\author{Daniel Ko}
\date{Spring 2020}
\begin{document}
\maketitle

Given positive integers $a$ and $b$, we want to compute some integers $s$ and $t$ such that
\[\operatorname{gcd}(a, b)=s a+t b\]
Consider the following iterative program LIN\_COMB $(a, b)$ which is supposed to accomplish this:\\
\hspace*{10mm}Initialize variables $c=a, d=b, s_{0}=1, t_{0}=0, s_{1}=0, t_{1}=1$\\
\hspace*{10mm}While $c \neq d,$ do the following:\\
\hspace*{20mm}If $c<d,$ then decrement $d$ by $c,$ decrement $s_{1}$ by $s_{0},$ decrement $t_{1}$ by $t_{0}$\\
\hspace*{20mm}Else if $c>d,$ then decrement $c$ by $d,$ decrement $s_{0}$ by $s_{1},$ decrement $t_{0}$ by $t_{1}$\\
\hspace*{10mm}Return $s_{0}$ and $t_{0}$\\

\section{6d}
Use the well-ordering principle and (c) to prove that LIN\_COMB satisfies
termination.
\begin{proof}
	We prove that LIN\_COMB terminates. Let $c_n$ and $d_n$ represent the values of $c$ and $d$ after $n$ iterations
	respectively. Consider the following two cases
	\begin{enumerate}[label=\arabic*.]
		\item{
		      $a = b$\\
		      \hspace*{10mm}The algorithm terminates because $a = b = c = d$.
		      }
		\item{
		      $a \neq b$\\
		      \hspace*{10mm}Let $c_n$ and $d_n$ represent the values of $c$ and $d$ after $n$ iterations
		      respectively. Suppose at the $n$th iteration, $c_n \neq d_n$.
		      Regardless of the values of $c_n$ and $d_n$, we can observe that $c_{n+1} + d_{n+1} \leq c_n + d_n - 1$.
		      This is because of the following statements. If $c_n > d_n$, then $c_{n+1} = c_n - d_n$.
		      If $c_n < d_n$, then $d_{n+1} = d_n - c_n$.
		      In (c) we have proved that $(c > 0) \land (d > 0)$ is a loop invariant.
		      This necessarily implies that $c_{n+1} \leq c_n - 1$ or $d_{n+1} \leq d_n - 1$. So in either case,
		      the sum of $c_{n+1}$ and $d_{n+1}$ will be at least one less the sum of $c_n$ and $d_n$.\\
			  \hspace*{10mm}By the well ordering
			  principle, since $c_n + d_n$ is strictly decreasing, there be an iteration $s$ such that 
			  $c_s + d_s$ is the smallest sum which is bounded by $(c > 0) \land (d > 0)$. 
			  When this is the case, $c_s = d_s$ because if $c_s \neq d_s$,
			  then you can once again decrement $c$ by $d_s$ or $d$ by $c_s$ and this will lead to a contradiction.
			  Thus, the algorithm terminates if $a \neq b$.
		      }
	\end{enumerate}
	Therefore, LIN\_COMB satisfies termination.

\end{proof}



\end{document}
