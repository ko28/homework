\documentclass[11pt]{scrartcl}
\usepackage[sexy]{../evan}
\usepackage{cmbright}
\usepackage{cancel}
\usepackage[T1]{fontenc}
%\usepackage{enumerate}
\usepackage[shortlabels]{enumitem}
\usepackage[utf8]{inputenc}
\usepackage[margin=1in]{geometry}
%\usepackage[pdfborder={0 0 0},colorlinks=true,citecolor=red{}]{hyperref}
\usepackage{amsmath}
\usepackage{amssymb}
\usepackage{setspace}
\usepackage{systeme}
%\usepackage[vlined]{algorithm2e}
\usepackage[linesnumbered,ruled,vlined]{algorithm2e}
\usepackage{mathtools}
\usepackage{graphicx}
\graphicspath{ {./images/} }
\SetStartEndCondition{ }{}{}%
\SetKwProg{Fn}{def}{\string:}{}
\SetKwFunction{Range}{range}%%
\SetKw{KwTo}{in}\SetKwFor{For}{for}{\string:}{}%
\SetKwIF{If}{ElseIf}{Else}{if}{:}{elif}{else:}{}%
\SetKwFor{While}{while}{:}{fintq}%
\newcommand{\forcond}{$i=0$ \KwTo $n$}
\SetKwFunction{FRecurs}{FnRecursive}%
\AlgoDontDisplayBlockMarkers\SetAlgoNoEnd\SetAlgoNoLine%


\usepackage{xcolor}
\DontPrintSemicolon

% Define pseudocode formatting
\renewcommand{\KwSty}[1]{\textnormal{\textcolor{blue!90!black}{\ttfamily\bfseries #1}}\unskip}
\renewcommand{\ArgSty}[1]{\textnormal{\ttfamily #1}\unskip}
\SetKwComment{Comment}{\color{green!50!black}// }{}
\renewcommand{\CommentSty}[1]{\textnormal{\ttfamily\color{green!50!black}#1}\unskip}
\newcommand{\assign}{\leftarrow}
\newcommand{\var}{\texttt}
\newcommand{\FuncCall}[2]{\texttt{\bfseries #1(#2)}}
\SetKwProg{Function}{function}{}{}
\renewcommand{\ProgSty}[1]{\texttt{\bfseries #1}}



\title{CS 577: HW 7}
\author{Daniel Ko}
\date{Summer 2020}

\begin{document}
\maketitle

\section{
  Prove that Strategic Advertising is NP-complete.
 }
Define a set $S$ to fulfill Strategic Advertising Problem (SAP) on $G$ with paths $\{P_i\}$,
if it is possible to place advertisements on at most $k$ of the nodes in $G$, so that each path $P_i$
includes at least one node containing an advertisement.

\subsection{
	Strategic Advertising is NP.
}
%,and either if the set $S$ .  
\begin{proof}
	Suppose we have a set $S$ that claims to fulfill SAP.
	We can iterate through all nodes in each $t$ paths, checking to see if a path contains a node
	that is also in the set $S$. Since, a path is at most $n$, where $n$ is the amount of nodes in the
	graph $G$, the computing complexity to check if a set $S$ fulfills SAP
	is $O(nkt)$ which is clearly polynomial.

\end{proof}

\subsection{
	Reduce Strategic Advertising to a known NP-complete problem.
}

\begin{proof}
	We claim that we can reduce Strategic Advertising to the Vertex Cover problem.
	We show that Vertex Cover $\leq_p$ Strategic Advertising.

	Suppose we have a graph $G=(V,E)$, and we want to find a vertex cover of size of at most $k$.
	We can convert this problem into a Strategic Advertising problem in polynomial time by doing the following.
	Since Strategic Advertising algorithm requires a directed graph, let $G^*=(V,E^*)$
	where $E^*$ is where we arbitrary set a direction to the edges in $E$. Generate paths,
	$P_i$, where each edge in $E^*$ is a path.
	We show that $G$ has a vertex cover of size of at most $k$ $\Leftrightarrow$
	there exists a set $S$ such that it fulfills SAP on $G^*$ with paths $\{P_i\}$.

	\begin{enumerate}[label=\roman*.]
		\item {
		      $G$ has a vertex cover of size of at most $k$ $\Rightarrow$
			  There exists a set $S$ such that it fulfills SAP on $G^*$ with paths $\{P_i\}$\\\-\\
			  Let $S$ be a set of nodes for a vertex cover for $G$.
			  $S$ will fulfill SAP because all edges will at least one endpoint in $S$ by the 
			  fact that $S$ is a vertex cover and since $\{P_i\}$ consists of all edges in $E$.
		      }
		\item {
		      There exists a set $S$ such that it fulfills SAP on $G^*$ with paths $\{P_i\}$
			  $\Rightarrow$ $G$ has a vertex cover of size of at most $k$\\\-\\
			  This means that we have most $k$ nodes in $S$ such that each path $P_i$ includes
			  such nodes. Notice that we since we defined $\{P_i\}$ to be the set of all edges in $E$,
			  $S$ will be a vertex cover for $G$.

		      }
	\end{enumerate}
	Thus, Vertex Cover $\leq_p$ Strategic Advertising as desired.
	Using Theorem 8.14, we conclude that Strategic Advertising is NP-complete.
\end{proof}






\pagebreak



\section{
  Prove that Scheduling is NP-complete.
 }
 We define the Scheduling problem to be the following:
 If you’re given a set of $n$ jobs with each specified by a set of time
 intervals. Is it possible to accept at least $k$ of them so that none of the accepted jobs
 overlap in time? 

\subsection{
	Scheduling is NP.
}
%,and either if the set $S$ .  
\begin{proof}
	Suppose we have a set $S$ that claims to fulfill the Scheduling problem. 
	We can clearly check if none of the jobs overlap by first sorting the intervals by 
	start time and iterate through $S$, checking if the end time of the $i$th job is equal or earlier than 
	the start time of the $i+1$th job. We also make sure that $|S| \geq k$. 
	Using an efficient sorting algorithm like merge sort, will take $O(n\log n)$, where $n$ is the number of 
	intervals given in the set, 
	and checking if any job overlap will take, $O(n)$. Hence, checking if $S$ fulfills Scheduling problem
	will take $O(n\log n)$, which clearly takes polynomial time. 
\end{proof}


\subsection{
	Reduce Scheduling to a known NP-complete problem.
}
\begin{proof}
	We claim that we can reduce Scheduling problem to the Independent Set problem.
	We show that Independent Set $\leq_p$ Scheduling.

	Suppose we have a graph $G=(V,E)$, and we want to find an independent set of at least $k$.
	We can convert this problem into a Scheduling problem in polynomial time by doing the following.
	Let where $e$ is the number of edges in our graph. 
	Let the available period of time that our jobs be from $(00:00)$ to $(\delta:00)$.
	Divide this period of time into $e$ intervals where each period represent an edge in $E$.
	%Let $e_i$ represent the interval that corresponds with the $i$th edge in $E$.
	Let $n$ be the number of vertices in $G$ and create a set with $n$ jobs with each job 
	corresponding to a vertex in $G$. For each job, set its intervals to the ones that represent its edges. 
	
	We show that $G$ has an independent set of at least $k$ if and only if there is at least $k$ jobs that do not overlap. 
	If there is independent set of size $k$, these nodes do not share any edges, thus the corresponding 
	time intervals for such jobs do not overlap. Conversely, if there is at least $k$ jobs that do not overlap, 
	the corresponding graph contains at least $k$ jobs that do not share edges with each other, which is an independent set.  
\end{proof}


Thus, Vertex Cover $\leq_p$ Strategic Advertising as desired.
Using Theorem 8.14, we conclude that Scheduling is NP-complete.







\end{document}

