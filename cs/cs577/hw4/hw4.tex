\documentclass[11pt]{scrartcl}
\usepackage[sexy]{../evan}
\usepackage{cmbright}
\usepackage{cancel}
\usepackage[T1]{fontenc}
%\usepackage{enumerate}
\usepackage[shortlabels]{enumitem}
\usepackage[utf8]{inputenc}
\usepackage[margin=1in]{geometry}
%\usepackage[pdfborder={0 0 0},colorlinks=true,citecolor=red{}]{hyperref}
\usepackage{amsmath}
\usepackage{amssymb}
\usepackage{setspace}
\usepackage{systeme}
\usepackage[vlined]{algorithm2e}
\usepackage{mathtools}
\SetStartEndCondition{ }{}{}%
\SetKwProg{Fn}{def}{\string:}{}
\SetKwFunction{Range}{range}%%
\SetKw{KwTo}{in}\SetKwFor{For}{for}{\string:}{}%
\SetKwIF{If}{ElseIf}{Else}{if}{:}{elif}{else:}{}%
\SetKwFor{While}{while}{:}{fintq}%
\newcommand{\forcond}{$i=0$ \KwTo $n$}
\SetKwFunction{FRecurs}{FnRecursive}%
\AlgoDontDisplayBlockMarkers\SetAlgoNoEnd\SetAlgoNoLine%


\title{CS 577: HW 4}
\author{Daniel Ko}
\date{Summer 2020}

\begin{document}
\maketitle

\section{
Given the random process described above, what is the expected number of
incoming links to node $v_j$ in the resulting network? Give an exact formula in terms of $n$ and $j$,
and also try to express this quantity asymptotically (via an expression without large summations)
using $\Theta$ notation
}

We know that $v_j$ can be only linked to nodes that come after it. 
For every node $v_i$, where $i > j$, the probability of $v_i$
connecting with $v_j$ is $\frac{1}{i-1}$, as there are $i-1$ nodes to choose from and $v_i$
connects to only one node. Hence, the expected number of
incoming links to node $v_j$ in the resulting network is the following:
$$
\sum_{i = j+1}^n \frac{1}{i-1}
$$
By Theorem 13.10 in the textbook we know that $H(\alpha) = \Theta(\log (\alpha))$,
\begin{align*}
	\sum_{i = j+1}^n \frac{1}{i-1} & = \sum_{i = j}^{n-1} \frac{1}{i} \\
	& = \sum_{i = 1}^{n-1} \frac{1}{i} - \sum_{i = 1}^{j-1} \frac{1}{i}\\
	& = H(n-1) - H(j-1)\\
	& = \Theta(\log(n-1)) - \Theta(\log(j-1))\\
	& =  \Theta(\log(n)) - \Theta(\log(j))\\
	& = \Theta(\log\Big(\frac{n}{j}\Big))
\end{align*}

\section{
Part (1) makes precise sense in which the nodes that arrive early carry an “unfair”
share of the connections in the network. Another way to quantify the imbalance is to observe
that, in a run of this random process, we expect many nodes to end up with no incoming links.
Give a formula for the expected number of nodes with no incoming links in a network grown
randomly according to this model.
}

Let $X_j$ be a random variable such that
$$
X_j = 
\begin{cases}
	1 & \textbf{if }  v_j \text{ has no incoming links} \\
	0 & \textbf{otherwise}
\end{cases}
$$
From part 1, we know that the probability for some $v_i$ to connect 
with $v_j$, where $i>j$, is $\frac{1}{i-1}$. It follows then that the probability that 
$v_i$ does not connect with $v_j$ is $1 - \frac{1}{i-1}$. To find the probability 
that no nodes connect to $v_j$, we multiply the probability of all $v_i$ not connecting 
to $v_j$. Hence the expected value of $X_j$ is
\begin{align*}
	E[X_j] & = \prod_{i = j+1}^n 1 - \frac{1}{i-1} \\
	& =  \Big( 1 - \frac{1}{j} \Big) \Big( 1 - \frac{1}{j+1} \Big) \cdots \Big( 1 - \frac{1}{n-1} \Big) \\
	& =  \Big( \frac{j-1}{j} \Big) \Big( \frac{j}{j+1} \Big) \cdots   \Big( \frac{n-2}{n-1} \Big)\\
	& = \frac{j-1}{n-1}
\end{align*}
assuming $n>1$.\\
To find the expected number of all nodes with no incoming links is to sum up 
the expected value of no incoming links for all nodes. Using the formula for triangle numbers,

\begin{align*}
	\sum_{j=1}^n E[X_j] & = \sum_{j=1}^n \frac{j-1}{n-1} \\
	& = \frac{1}{n-1} \Big( \sum_{j=1}^n j - \sum_{j=1}^n 1 \Big) \\
	& = \frac{1}{n-1}  \Big( \frac{n(n+1)}{2} - n \Big)\\
	& = \frac{1}{n-1}  \Big( \frac{n(n-1)}{2}  \Big)\\
	& = \frac{n}{2}
\end{align*}
\end{document}

